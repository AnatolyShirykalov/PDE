\section{17 ноября 2014}
Сейчас мы напишем формулу для нахождения решения задачи Коши. В прошлый раз мы взяли решение и обрезали его, оставив только положительные значения времени. Сейчас попробуем воспользоваться теоремой существования.

Как строить фундаментальные решения я уже говорил и даже пример привёл.
\begin{The}
  Пусть $w$ "--- решение задачи Коши 
\begin{equation}\label{zKoshi}
  \begin{cases}
     \L w = f(x),\\
     w(0)=w_0,\\
     \dotfill\\
     w^{(m-1)}(0)=w_{m-1}.
  \end{cases}
\end{equation} где
\[
  \L = \DP{^m}{x^m} + a_{m-1}\DP{^{m-1}}{x^{m-1}}+\dots+a_1\DP{}x+a_0,\quad f\in C^\infty(\R),\ a_s\in\R,\ s=0,1,2\dots,m-1,\quad a_n=1.
\]
Тогда
\begin{equation}\label{stars2}
  w(x) = \RY s1m\RY p0{s-1} a_sw_p W^{(s-p-1)}(x) + \int\limits_0^x W(x-\xi)f(\xi) \,d\xi,
\end{equation}
где $W$ "--- решение задачи Коши вот такой
\[
  \begin{cases}\L W = 0,\\ W(0)=\dots=W^{(m-2)}(0)=0,\\ W^{(m-1)}(0)=1.\end{cases}
\]
\end{The}

Вот такой формулой достаточно удобно пользоваться.

\begin{Proof}
  Решение существует по теореме существования и единственноси. Пусть $w$ "--- решение задачи Коши \eqref{zKoshi}. Обозначим $\Til w(x) = \theta(x)w(x)$, где
\[
  \theta(x) = \begin{cases} 1,&x\ge0;\\0,&x<0.\end{cases}
\]
И $\Til f(x):=\theta(x)f(x)$. 
\[
  \L \Til w = \RY s1m\RY p0{s-1} a_s w_p\delta^{(s-p-1)}(x) + \Til f(x).
\]
Тогда по теореме единственности будем иметь \[\Til w(x)=\RY s1m\RY p0{s-1} a_sw_p\delta^{(s-p-1)}(x)\star\E(x)+f(x)\star\E(x),\] где $\E(x) = \theta(x) W(x)$ "--- фундаментальное решение оператора $J$.

Мы знаем, что у $\delta$-функции с любой обобщённой свёртка существует и равна
\[
  \delta^{(s-p-1}(x) \star \E(x) = \left(\DP{}x\right)^{s-p-1}\big(\delta(x)\star\E(x)\big) = \left(\DP{}x\right)^{s-p-1}\E(x) = \theta(x) W^{(s-p-1)}(x),\]
так как $\E(x),\E(x),\dots,\E^{(m-2)}(x)$ абсолютно непрерывны.
Для чего придуманы абсолютно непрерывные функции: это в точности те функции, для которых верна формула Ньютона"--~Лейбница.

Мы заметим вот что. Функции $\Til f$ и $\E(x)$ локально интегрируемы по Лебегу. Значит, свёртку можно считать по классической формуле.
\[
  \Til f\star \E(x) = \int\limits_{-\infty}^{+\infty}\E(x-\xi)\Til f(\xi)\,d\xi = 
\]
Теперь вспоминаем, что такое $\E$:
\[
  = \FI\theta(x-\xi) W(x-\xi)\theta(\xi) f(\xi)\,d\xi = \theta(x)\int\limits_x^x W(x-\xi)f(\xi)\,d\xi.
\]
$\theta(x)$ для того, чтобы в отрицательных точках точно был ноль.
\[
  \Til w(x) = \theta(x)\RY s1m\RY p0{s-1} a_sw_p W^{(s-p-1)}(x) + \theta(x)\int\limits_0^x W(x-\xi)f(\xi)\,d\xi.
\]
Если $x>0$, то последнее выражение совпадает с \eqref{stars2}. А как с отрицательными? Покажем, что \eqref{stars2} имеет место и при $x<0$. В задаче \eqref{zKoshi} сделаем замену переменных $x\rightleftharpoons -x$. Получим для $v(x) = (-1)^m w(-x)$ следующую задачу Коши
\begin{equation}\label{stars3}
  \begin{cases}
    (-1)^m\L^* v = f(-x),\\
    v^{(P)}(0) = (-1)^{p+m}w_p,&p=0,1,2,\dots,m-1,
  \end{cases}
\end{equation}
где $\Til L^* = (-1)^m\DP{^m}{x^m} + (-1)^{m-1}a_{m-1}\DP{^{m-1}}{x^{m-1}}+\dots-a_1\DP{}x+a_0$ "--- оператор, формально сопряжённый к $\L$. Откуда это берётся? Вот была задача для функции $w$ $\L w = f(x)$. Теперь мы рассмотрели новую функцию $v(x)$. Что можно написать: $w(x) = (-1)^m v(-x)$, подставляем в задачу, что получем (надо дифференцировать функцию $v$, а когда мы дифференцируем сложную функцию, возникает $(-1)$ в той же степени, что и производная, а если применять оператор, формально сопряжённый, то все эти $(-1)$ сокращаются)
\[
  (-1)^m\L^* v(x) = \cancel{(-1)^m} \RY s1m(-1)^s a_s\DP{^{s}}{x^s}\cancel{(-1)^m}w(-x)=
\]
Я должен $s$ раз профифференцировать функцию $\DP{^s w(-x)}{x^s} = \left.(-1)^s \DP{^sw(y)}{y^s}\right|_{y=-x}$
\[
  = \RY s0m a_s\DP{^sw(y)}{y^s}\bigg|_{y=-x} = f(-x).
\]
Вот и получили уравнение. А как найти условие? Это ещё проще.
\[
  v^{(p)}(0) = (-1)^m\DP{^p}{x^p} w(-x)\bigg|_{x=0} = (-1)^{m+p} w_p.
\]
Собственно говоря, вот \eqref{stars3} и вытекает.

Очевидно, что функция $V(x) = (-1)^{m-1} W(-x)$ (это $W$ из \eqref{stars2}) будет решением задачи Коши 
\[
  \begin{cases}
    (-1)^m\L^* V = 0,\\
    v(0)=\dots=v^{m-2}(0) = 0,\\
    v^{m-1}(0)=1.
  \end{cases}
\]
Таким образом, функция $\theta(x) V(x)$ является фундаментальным решением оператора $(-1)^m\L^*$. Повторяя предыдущие рассуждения с заменой функции $\Til w(x)$ на $\Til v(x) = \theta(x) v(x)$ получим 
\[
  \Til v(x) =  \theta(x)\RY s1m\RY p0{s-1}(-1)^{s+p}a_s w_p V^{(s-p-1)}(x) + \theta(x)\int\limits_0^x V(x-\xi)f(-\xi)\,d\xi.
\]
При $x>0$ из последнего выражения находим 
\[
  (-1)^mw(-x) = \RY s1m\RY p0{s-1}(-1)^{s+p}a_s w_p V^{s-p-1}(x) + \int\limits_0^x(x-\xi) f(-\xi)\,d\xi.
\]
Теперь надо перейти к $W$. Под интегралом сделаем замену $\zeta = -\xi$ и получим \eqref{stars2} для отрицательных аргументов функции $w$.
\end{Proof}

\subsection{Фундаментальное решение оператора Лапласа}
\begin{The}
 Пусть $u\in C(\ol\W)\cap C^1(\W)$, где $\W$ "--- ограниченная область с кусочно гладкой границей. Тогда
\[
  \Gint\W\CP f{x_i}\,dx = \Gint{\dl\W} f \cos(\boldsymbol v,x_i)\,ds,
\]
где $\boldsymbol v$ "--- вектор единичной нормали к $\dl\W$, внешней по отношению к $\W$.
\end{The}
Эта формула называется формулой Грина.
\begin{Proof}
  В общей формуле Стокса (см. Диф. геом.)
\[
  \Gint{\dl\W}\w = \Gint\W\,d\w
\]
положим $\w = (-1)^{i-1}f(x)dx^1\wedge\dots\wedge dx^{i-1}\wedge dx^{i+1}\wedge\dots\wedge dx^m$.
\end{Proof}
Здесь происходит состыковка строгой математики и нестрогой. Если нужно что-то посчитать, нагляднее пользоваться косинусами. Если же надо что-то доказывать, то пользуемся дифференциальной геометрией.

Из этой теоемы есть замечательное следствие, а именно формула Грина интегрирования по частям
\begin{The}
  Пусть $f,g\in C(\ol\W)\cap C^1(\W)$, $\W$ "--- ограниченная область с кусочно гладкой границей. Тогда
\[
  \Gint\W\CP f{x_i}g\,dx = \Gint{\dl\W} fg\cos(\boldsymbol \nu x_i)\,ds - \Gint\W f\CP g{x_i}\,dx,
\]
где $\boldsymbol \mu$ "--- вектор единичной нормали к $\dl\W$, внешней по отношению к области $\W$.
\end{The}
\begin{Proof}
  Берём в предыдущей теореме $u = fg$, получаем
\[
  \Gint\W\CP f{x_i}\,dx +\Gint\W f\CP g{x_i}\,dx = \Gint{\dl\W}fg\cos(\boldsymbol\nu,x_i)\,ds.
\]
\end{Proof}
