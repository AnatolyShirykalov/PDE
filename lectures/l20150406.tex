\section{Формулы Даламбера, Пуассона и прочие}

Мы доказываем существование и единственность классического решения Задачи Коши. Чуть посложнее чем формула Даламбера, получим формулу Пуассона. Это "--- случай $n=2$.
Фундаментальное решение волнового оператора в случае $n=2$ имеет вид
\[
  \E_2(x,t) = \frac{\theta(at - |x|}{2\pi a\sqrt{a^2t^2-|x|^2}}.
\]
Это локально интегрируемая функция. Раз так, все свёртки можно считать, как классические свёртки.
\begin{multline*}
  \Til f(x,t)\star\E_2(x,t) = \int\limits_{-\infty}^{+\infty} \Gint{\R^2} \E_2(x-y,t-\theta) \Til f(y,\theta)\,dy\,d\theta = \\=
  \int\limits_{-\infty}^{+\infty} \Gint{\R^2}\frac{\theta\big(a(t-\theta) -|x-y|\big)\Til f(y,\theta)\,dy\,d\theta }{2\pi a\sqrt{a^2(t-\theta)^2 - |x-y|^2}} = 
  \frac1{2\pi a}\int\limits_0^{+\infty}\Gint{B_{at}^x} \frac{f(y,\theta)\,dy\,d\theta}{\sqrt{a^2(t-\theta)^2 - |x-y|^2}}.
\end{multline*}

С первым слагаемым мы разобрались. А как быть со вторым? Мы уже брали компактные исчерпания единицы, мы сейчас результат напишем
\[
  \forall\ \phi\in \D(\R^3)\pau 
  \big(\E_2(x,t)\star u_1(x)\delta(t),\phi(x,t)\big) =
  \Big(\E_2(x,t) u_1(y)\delta(\theta),
    \lambda \big(^2t^2 - |x|^2\big)\lambda(t)\lambda(\theta)\phi(x+y,t+\theta)\Big) = 
\]
% рисунок $\lambda$
Здесь $\lambda\in C^{\infty}(\R)$, $\lambda|_{(-\infty,-1)}=0$, $\lambda|_{[-1/2,+\infty)}=1$.
\begin{multline*}
= \bigg(\E(x,t),\Big(u_1(x),\delta(t),\lambda\big(a^2t^2-|x|^2\big)\lambda(t)\lambda(\theta)\phi(x+y,t+\theta)\Big)\bigg) =\\
= \Bigg(\E(x,t),\bigg(u_1(x),\Big(\delta(t),\lambda\big(a^2t^2-|x|^2\big)\lambda(t)\lambda(\theta)\phi(x+y,t+\theta)\Big)\bigg)\Bigg) =\\
= \bigg(\E_2(x,t),\Big(u_1(y),\lambda\big(a^2t^2 - |x|^2\big)\lambda(t)\phi(x+y,t)\Big)\bigg) = \\
= \bigg(\E_2(x,t),\Gint{\R^2} u_1(y)\lambda\big(a^2t^2 - |x|^2\big)\lambda(t)\phi(x+y,t)\,dy\bigg) = \\
= \frac1{2\pi a}\int\limits_{-\infty}^{+\infty}\Gint{\R^2}\Gint{\R^2}
  \frac{\theta\big(at - |x|\big)}{\sqrt{a^2t^2-|x|^2}}\times
    u_1(y)\lambda\big(a^2t^2 - |x|^2\big)\lambda(t)\phi(\underbrace{x+y}_{\xi},t)\,dy\,dx\,dt=\\
= \frac1{2\pi a}\int\limits_{-\infty}^{+\infty}\Gint{\xi\in\R^2}\Gint{B^\xi_{at}}
  \frac{\theta\big(at - |\xi-y|\big)}{\sqrt{a^2t^2 - |\xi-y|^2}}u_1(y)\lambda\big(a^2t^2 - |\xi-y|^2\big)\lambda(t)\phi(\xi,t)\,dy\,d\xi\,dt=\\
= \frac1{2\pi a}\int\limits_{-\infty}^{+\infty}\Gint{\R^2}\Gint{\R^2}
  \frac{\theta(t)u_1(y)}{\sqrt{a^2t^2 - |\xi-y|^2}}\phi(\xi,t)\,dy\,dt=
  \bigg(\frac{\theta(t)}{2\pi a}\Gint{B^\xi_{at}}
	\frac{u_1(y)}{\sqrt{a^2t^2 - |\xi-y|^2}}\,dy,\phi(\xi,t)\bigg).
\end{multline*}

Другими словами, 
\[
\E_2(x,t)\star u_1(x,t)\delta(t) = 
  \frac{\theta(t)}{2\pi a}\Gint{B^\xi_{at}} \frac{u_1(\xi)\,d\xi}{\sqrt{a^2t^2 - |x-\xi|^2}}.
\]

Это мы получили второе слагаемое. Но есть ещё третье, которое достаточно тривиально
\[
  \E_2(x,t)\star u_0(x)\delta'(t) = 
  \CP{ }t\big(\E_2(x,t)\star u_0(x)\delta(t)\big) = \CP{ }t\frac{\theta(t)}{2\pi a}\Gint{B^x_{at}}
	\frac{u_0(\xi,t)\,d\xi}{\sqrt{a^2t^2-|x-\xi|^2}}.
\]

Осталось показать, что эти решения дают достаточно гладкие функции. 
\begin{multline*}
 \frac1{2\pi a}\int\limits_0^t\Gint{B^x_{a(t-\tau)}}\frac{f(\xi,\tau)\,d\xi\,d\tau}{\sqrt{a^2(t-\tau)^2 - |x-\xi|^2}} = 
 \left\{
  \begin{matrix}
    \xi\in B_{a(t-\tau)}^x\iff |\xi-x|<a(t-\tau);\\
    y = \frac{\xi -x}{t-\tau}\pau |y|<a;\\
   d\xi = (t-\tau)^2\,dy.
\end{matrix}
\right\}=\\
=\frac{(t-\tau)^2}{2\pi a}\int\limits_0^t \Gint{B_a}\frac{f\big(x+(t-\tau)y,\tau\big)(t-\tau)^2\,dy\,d\tau}{\sqrt{a^2(t-\tau)^2-(t-\tau)^2|y|^2}} 
= \frac1{2\pi a}\int\limits_0^t\Gint{B_a} \frac{f\big(x+(t-\tau)y,\tau\big)(t-\tau)\,dy\,d\tau}{\sqrt{a^2-|y|^2}}.
\end{multline*}

Корень квадратный "--- это хорошая особенность и она суммируется. Можете и по $t$ и $x$ непрерывно дифференцировать два раза до замыкания. Мы доказали, что это всё принадлежит вот такому классу $C^2\big(\R^2\times[0,+\infty)\big)$. Это более сильное утверждение, чем нам требуется.

Аналогично
\begin{multline*}
  \frac1{2\pi a} \Gint{B^x_{at}}\frac{u_1(\xi)\,d\xi}{\sqrt{a^2t^2-|x-\xi|^2}} = 
  \left\{
   \begin{matrix}
     \xi\in B^x_{at}\iff |\xi -x|<at;\\
     y = \frac{\xi -x}{t};\\
     \xi = x + ty;\\
    d\xi = t^2 dy.
\end{matrix}
  \right\} = \\
 = \frac1{2\pi a} \Gint{B_a}\frac{u_1(x+ty)t^2\,dy}{\sqrt{a^2t^2 - t^2|y|^2}} = 
   \frac t{2\pi a} \Gint{B_a}\frac{u_1(x+ty)}{\sqrt{a^2 - |y|^2}}\in C^2\big(\R^2\times[0,+\infty)\big).
\end{multline*}

Дифференцируем по времени
\[
  \CP{ }t\frac1{2\pi a}\Gint{B_{at}^x} \frac{u_0(\xi)d\xi}{\sqrt{a^2t^2 - |x-\xi|^2}} = 
  \left\{\begin{matrix}
    y = \frac{\xi -x}t;\\
    \xi = x+ty;\\
    d\xi = t^2\,dy.
  \end{matrix}
  \right\} = 
\frac1{2\pi a}\CP{ }t\Gint{B_a} \frac{t u_0(x+ty)\,dy}{\sqrt{a^2t^2-|y|^2}}\in C^2\big(\R^2\times[0,+\infty)\big).
\]
