\section*{Уравнения в частных производных}
\section{6 октября 2014}
\subsection{Прямое произведение двух обобщённых функций}
В прошлый раз раз мы с вами обсуждали теорему о конечном порядке сингулярности.

Нам понадобится ещё одно понятие: прямое произведение двух функций. Ещё иногда его называют тензорным произведением. Вот тут мы не про тензоры говорим, а про функции "--- это такой вырожденный тензор. В чём смысл прямого произведения: берёте функцию $\R^n\to \C$, другую $\R^m\to \C$, перемножаете, получаете функцию $m+n$ переменных.

Пусть $f\in\C^\infty(F)$, $g\in C^\infty(G)$, где $F\subset \R^n$ "--- открытое непустое множество, $G\subset \R^m$ "--- открытое непустое множество. Тогда просто построим вот такую вот функцию:
\[(\ve xn,\ve[n+1]x{m+n})\mapsto f(\ve xn)\cdot g(\ve[n+1]x{n+m}).\]
Называем её прямым произведением $f$ и $g$. Как мы её обозначаем, чтобы показать, что это всё-таки что-то новое: $fg$, $f\cdot g$ или $f\otimes g$. Это $ F\times G \to \C$.

Мы знаем, что $C^{\infty}$ вкладывается в $D'$. Так что ничего нового не должно происходить. Будем понимать $f\in C^\infty (F)$ и $g\in C^\infty (G)$ как обобщённые функции из $D'(F)$ и $D'(G)$. Что это означает? Возьмём прямое произведение $fg\in C^\infty(F\times G)$ и применим к $\phi\in D(F\times G)$.
\[
  \big(f(x)g(y),\phi(x,y)\big) = \int\limits_{F\times G} f(x)g(y) \phi(x,y) \,dx\,dy = \int\limits_F\int\limits_G f(x)g(y) \phi(x,y) \,dy\,dx = \Big(f(x),\big(g(y),\phi(x,y)\big)\Big).
\]
Это на самом деле уже готовая формула для определения прямого произведения.

\begin{Def}\label{DefOtimes}
  Прямым произведением обобщённых функций $f(x)\in D'(F)$ и $g(y)\in D'(G)$ называется обобщённая функция $f(x)g(y)$ (сохраняем символ аргумента, хотя понимаем, что это не бесконечно гладкие функции), такая, что
\[
 \forall\ \phi(x,y)\in D(F\times G)\pau \big(f(x)g(y),\phi(x,y)\big) = \Big(f(x),\big(g(y),\phi(x,y)\big)\Big).
\]
\end{Def}

При каждом $x$ сначала получаем $\big(g(y),\phi(x,y)\big)$. Понятно, что это бесконечно гладкая функция, понятно, что у неё гладкий носитель; по теореме из конца прошлой лекции знаем, что $\big(g(y),\phi(x,y)\big)\in D(F)$. Непонятно, почему при этом получится непрерывная в севенциальном смысле функция; линейность действительна, нет сомнений. Нужно же показать, что если $\phi_k\to \phi$, то всё вот это $\big(f(x)g(y),\phi_k(x,y)\big)\to \big(f(x)g(y),\phi(x,y)\big)$. Не знаю, насколько для вас это очевидно, поэтому давайте на этом остановимся.

Итак, очевидно, что $f(x)g(y)$ "--- линейная функция. Покажем, что $f(x)g(y)$ непрерывный в $D(F\times G)$ функционал. В самом деле, пусть $\phi_k(x,y)\to \phi(x,y)$ в $D(F\times G)$ при $k\to \infty$, то есть
\begin{roItems}
  \item Существует компакт $H\subset F\times G\colon \forall\ k\in\N\pau \supp \phi_k\subset H$;
  \item $\forall\ m\pau \|\phi_k-\phi\|_{C^m(F\times G)}\to 0$ при $k\to \infty$.
\end{roItems}
Тогда $\big(g(y),\phi_k(\cdot,y)\big)\to \big(g(y),\phi(\cdot,y)\big)$ в норме $C^m(F)$ при $k\to\infty$ для всех $m\ge 0$.

Другими словами, для любого мультииндекса $\alpha = (\alpha_1,\dots,\alpha_n)$ (я уже могу дифференцировать по параметру)
\[\p_x^\alpha\big(g(y),\phi_k(x,y)\big)\to \p_x^\alpha \big(g(y),\phi(x,y)\big)\]
равномерно по $x\in F$.

Вот теперь можем дифференцировать по параметру. Имеем,
\[\p_x^\alpha \big(g(y),\phi_k(x,y)\big) = \big(g(y),\p_x^\alpha\phi_k(x,y)\big),\qquad
  \p_x^\alpha \big(g(y),\phi(x,y)\big) = \big(g(y),\p_x^\alpha\phi(x,y)\big).
\]
Ну а теперь возьмём разность.
\[\Big|\p_x^\alpha \big(g(y),\phi_k(x,y)\big) - \p_x^\alpha \big(g(y),\phi(x,y)\big)\Big| = 
  \Big|\big(g(y),\p_x^\alpha\phi_k(x,y) - \p_x^\alpha\phi(x,y)\big)\Big|.
\]
Дальше мы знаем, что носитель разности $\p_x^\alpha\phi_k(x,y) - \p_x^\alpha\phi(x,y)$ лежит в некотором компакте. Поэтому я могу записать, что всё это равно
\[\bigg| \Big(g(y),\eta(y)\big(\p_x^\alpha\phi_k(x,y) - \p_x^\alpha\phi(x,y)\big)\Big)\bigg|,\]
где $\eta\in D(G)$ и при этом $\eta\equiv1$ в окрестности проекции $H$ на $G$. Ничего не изменилось от такого домножения, но мне это даёт возможность написать следующее. Таким образом,
\[
  \Big|\p_x^\alpha \big(g(y),\phi_k(x,y)\big) - \p_x^\alpha \big(g(y),\phi(x,y)\big)\Big| =
  \Big| \big(g(y)\eta(y),\p_x^\alpha\phi_k(x,y) - \p_x^\alpha\phi(x,y)\big)\Big|,
\]
где $g(y)\eta(y)$ "--- обобщённая функция с компактным носителем. А раз так, то я могу написать, что всё это не превосходит
\[
  \le A\big\|\p_x^\alpha \phi_k(x,\cdot) -\p_x^\alpha\phi(x,\cdot)\big\|_{C^N(G)}\le A\|\phi_k-\phi\|_{C^{N+|\alpha|}(F\times G)}\to 0.\pau (k\to \infty.)
\]

Дальше я должен показать, что есть компакт, где лежат все носители. Очевидно, что $\supp\big(g(y),\phi_k(\cdot,y)\big)$ есть подмножество проекции $H$ на $F$.

Тем самым, $\big(g(y),\phi_k(\cdot,y)\big)\to \big(g(y),\phi(\cdot,y)\big)$ в $D(F)$ при $k\to \infty$, поэтому ввиду непрерывности $f(x)$ будем иметь
\[\Big(f(x),\big(g(y),\phi_k(x,y)\big)\Big)\to \Big(f(x),\big(g(y),\phi(x,y)\big)\Big),\]
то есть $f(x)g(y)$ непрерывный функционал на $D(F\times G)$.

Всё очень просто. Надо только воспользоваться неравенством из компактности носителя.
%%%%%%%%%%%%%%%

\subsection{Коммутативность прямого произведения обобщённых функций}
Когда писали прямое произведение двух бесконечно гладких функций, воспользовались теоремой Фубини. Могли бы интегрировать в другом порядке и результат интегрирования не изменился. А для обобщённых функций порядок важен?
\begin{The}\label{kompramprod}
  Пусть $f(x)\in D'(F)$, $g(y)\in D'(G)$, а $\phi\in D(F\times G)$. Тогда 
\[\Big(f(x),\big(g(y),\phi(x,y)\big)\Big) = \Big(g(x),\big(f(y),\phi(x,y)\big)\Big).\]
\end{The}

Вообще прямое произведение не коммутативно по своей природе. А у нас получилось коммутативное "--- частный случай. У нас в определении прямого произведения обобщённых функций есть некоторый произвол "--- вот о чём теорема.

\begin{Proof}
  Если бы $\phi(x,y) = \psi_1(x)\cdot \psi_2(y)$, всё очевидно. А как в общем случае сделать? Любую функцию $\phi$ разложим, так сказать, в ряд. Любую функцию $\phi\in D(F\times G)$ можно разложить в ряд
\[\phi(x,y) = \ry\phi k(x)\phi_k(y),\]
где $\phi_k\in D(F)$, $\phi_k\in D(G)$, $k=1,2,\ldots$

Так всегда можно сделать. Надо только понять как. Мне приходит в голову разложить в ряд фурье, по каким-то экспонентам разложить. Коэффициенты очень быстро сходится к нулю у бесконечно гладких функций. Мне нужна сходимость $C^N(F\times G)$. Коэффициенты убывают быстрее любой степени, получу всё как надо. Идея понятна, а сейчас я это напишу.

\begin{enumerate}
\item Разложим $\phi(x,y)$ в ряд Фурье
\[
\phi(x,y) = \sum\limits_{\substack{s\in\Z^n\\ p\in\Z^m}}c_{sp} e^{isx}e^{ipy}.
\]

Мы можем выбрать достаточно большой куб размерности $m+n$, которому будет принадлежать $F\times G$. Разнесём этот куб по $\R^{m+n}$, доопределив $\phi$ до периодической. Нам хочется, чтобы ребро куба было $2\pi$. Поэтому делаем такую приписку. Считаем без ограничения общности $F\times G\subset (-\pi,\pi)^{n+m}$.

Ввиду того, что $\phi$ бесконечно гладкая функция, будем иметь
\[\sum\limits_{\substack{s\in\Z^n\\ p\in\Z^m}} |c_{sp}|(1+|s|+|p|)^N<+\infty\pau \forall\ N\in\N.\]

\item  Возьмём $\eta\in D(F) $\footnote{$D$ и $C_0^{\infty}$, кстати, одно и то же.}, $\lambda\in D(G)$, такие, что $\eta\equiv 1$ на проекции $\supp \phi$ на $F$, $\lambda\equiv 1$ на проекции $\supp \phi$ на $G$. Тогда, очевидно, получим
\[ \eta(x)\lambda(y)\phi(x,y)\equiv \phi(x,y).\]
%можно нарисовать картинку, как носитель и его прокции выглядят в R^2
При этом весь ряд можно написать вот так вот, смотрите (слева и справа умножили на бесконечно гладкую функцию)
\[
  \phi(x,y) = \sum\limits_{\substack{s\in\Z^n\\ p\in\Z^m}}c_{sp} e^{isx}\eta(x)e^{ipy}\lambda(y).
\]
Заметим, что ряд $\sum\limits_{\substack{s\in\Z^n\\ p\in\Z^m}}c_{sp} e^{isx}e^{ipy}$ сходится к функции $\phi(x,y)$ в пространстве $C^N(F\times G)$ для любого $N$ ввиду бесконечной гладкости $\phi$. Таким образом, $\sum\limits_{\substack{s\in\Z^n\\ p\in\Z^m}}c_{sp} e^{isx}\eta(x)e^{ipy}\lambda(y)$ сходится к $\phi\equiv \eta(x)\lambda(y)\phi$ в пространстве $C^N(F\times G)$ для любого $N$. Ну и всё, сходимость у нас есть, какая нужно.

Тем самым вот этот вот ряд $\sum\limits_{\substack{s\in\Z^n\\ p\in\Z^m}}c_{sp} e^{isx}\eta(x)e^{ipy}\lambda(y)$ сходится к $\phi(x,y)$ в пространстве $D(F\times G)$ (частичные суммы ряда образуют сходящуюся последовательность). Перенумеруем члены этого ряда. Получим
\[\phi(x,y) = \ry\phi k(x)\phi_k(y),\pau \phi_k(x)\rightleftharpoons e^{isx}\eta(x) c_{sp},\pau \psi_k(y) \rightleftharpoons e^{ipy}\lambda(y).\]
Имеем
\[\Big(f(x),\big(g(y),\RY k1N \phi_k\psi_k\big)\Big) = \RY k1N\big(f(x),\phi_k(x)\big)\big(g(y),\psi_k(y)\big).
\]
Аналогично 
\[\Big(g(y),\big(f(x),\RY k1N \phi_k\psi_k\big)\Big) = \RY k1N\big(g(y),\psi_k(y)\big)\big(f(x),\phi_k(x)\big).
\]

То есть для всех $k$
\[ \Big(f(x),\big(g(y),\RY k1N \phi_k\psi_k\big)\Big) = \Big(g(y),\big(f(x),\RY k1N \phi_k\psi_k\big)\Big).\]
Переходя к пределу при $N\to\infty$ получим
\[\Big(f(x),\big(g(y),\phi(x,y)\big)\Big) = \Big(g(x),\big(f(y),\phi(x,y)\big)\Big).\]
\end{enumerate}
\end{Proof}

\begin{Task}
 Почему предел существует? (Из непрерывности прямого произведения.)
\end{Task}

\section{13 октября 2014}
На чём мы остановились. Доказали теорему о конечном порядке сингулярности обобщённой функции с компактном носителем. Доказали коммутативность прямого произведения.

\subsection{Свёртка обобщённой функции}
Я напомню, что такое свёртка двух функций в пространстве $L_1(\in\R^n)$. Пусть $f,g\in L_1(\R^n)$ (пока просто считайте, что это функции, интегрируемые по Риману). Что называется свёрткой?
\[ f \star g (x) = \int\limits_{\R^n} f(\underbrace{x-y}_{\xi})g(\underbrace{y}_{x-\xi})\,dy = \int\limits_{\R^n}g(x-\xi)f(\xi)\,d\xi.\]
Я всегда забываю $x-y$ или $y-x$, но вот есть способ себя проверить: $f\star g = f\star y$.

Что замечательно в пространстве $L_1$? Это свёрточная алгебра. Там естественно есть структура линейного пространства.

Давайте попробуем посчитать (есть такая теорема, называется Фубини)
\[\Gint{\R^n}
=\big|f\star g(x)\big|\,dx = \Gint{\R^n}dx\bigg|\Gint{\R^n}f(x-y)g(y)\,dy\bigg|\le \Gint{\R^n}dx\Gint{\R^n}\big|f(x-y)\big|\big|g(y)\big|\,dy.\]

Как работают с интегралом Лебега. Рассматривают фундаментальные последовательности ступенчатых функций. Они сходятся к измеримым функциям. Норма опредяется так: берёте интеграл Лебега от модуля функции. Существует предел интегралов ступенчатых функций, этот предел называют интегралом Лебега. Что я хочу сказать: если функция интегрируема по Лебегу, то её модуль интегрируем, поэтому наш интеграл слева существует. И есть такая теорема, что модуль интеграла не превосходит интеграла модуля.

А что такое теорема Фубини: есть функция двух векторных переменных, тогда можно кратный интеграл считать как повторный. Что у нас получается:
\[
  = \Gint{\R^n}dy\big|g(y)\big|\underbrace{\Gint{\R^n}\big|f(x-y)\big|\,dx}_{\Gint{\R^n} \big|f(\xi)\big|\,d\xi }= \Gint{\R^n}\big|g(y)\big|,dy  \Gint{\R^n}\big|f(\xi)\big|\,d\xi = \|f\|_{L_1(\R^n)}\|g\|_{L_1(\R^n)}.
\]

Тем самым мы утверждаем, что
\[\|f\star g\|_{L_1(\R^n)}\le \|f\|_{L_1(\R^n)} \|g\|_{L_1(\R^n)}.\]
И интеграл от свёртки поэтому существует. Мы сразу двух зайцев убили: из теоремы Фубини показали и что свёртка существует и что она суммируема.

Как поступать с обобщёнными функциями? Мы знаем, что $L_1(\R^n)$ вкладывается в $D'(\R^n)$. Будем понимать $f,g\in L_1(\R^n)$ как обобщённые функции из $D'(\R^n)$.
\[\begin{cases}f\colon \phi\mapsto \Gint{\R^n} f\phi\,dx,& \phi\in D(\R^n);\\
  g\colon \phi\mapsto \Gint{\R^n} g\phi\,dx,& \phi\in D(\R^n);\end{cases}\]
При этом
\[f\star g\colon \phi\mapsto \Gint{\R^n} f\star g \phi\,dx,\qquad \phi\in D(\R^n).\]

Что бы нам сделать, чтобы угадать определение для обобщённой функции? Делать какие-то замены.
\[\Gint{\R^n} f\star g(x)\phi(x)\,dx = \Gint{\R^n}\Gint{\R^n} f(x-y)g(y)\phi(x)\,dy\,dx = \Gint{\R^n}\Gint{\R^n} f(\xi)g(y)\phi(\xi+y)\,dy\,d\xi.\]
Последнее похоже на действите прямого произведение на функцию, но это не так. Всё же $\phi(\xi+y)\in C^\infty(\R^{2n})$, но не имеет компактного носителя. Действительно $\supp \phi\subset [-A,A]$, значит, $-A\le \xi+y\le A$ "--- диагональная полоса, это не компактное множество.

Как нам, собственно говоря, выйти из положения? Введём понятие.
\begin{Def}
  Компактным исчерпанием единицы в $\R^m$ называется последовательность функций 
\[\eta_k\in D(\R^m),\ k=1,2,\dots,\]
таких, что выполнены два свойства:
\begin{roItems}
  \item Для любого компакта $H\subset \R^m\pau \exists\ k_0\colon \forall\ k>k_0\pau \eta_k\big|_H=1$;
  \item $\forall\ s\pau \exists A\colon \forall\ k\pau \|\eta_k\|_{C^s(\R^n)}\le A$.
\end{roItems}
\end{Def}

Ествественный вопрос: существует ли хоть одна такая последовательность? Оказывается существует. Пусть $\eta\in D(B_1)$, такая, что $\eta\big|_{B_{\frac12}} = 1$. Положим
\[\eta_k(x) = \eta\left(\frac xk\right),\pau k=1,2,\dots\]
Первое свойство, очевидно, выполнено. Второе:
\[\p_x^\alpha\eta_k(x) = \frac1{k^{|\alpha|}}\p_\xi^\alpha\eta(x)\big|_{\xi = \frac xk},\pau \alpha = (\alpha_1,\dots,\alpha_n),\ |\alpha| =\alpha_1+\ldots+\alpha_n.\]

А теперь мы готовы написать определение свёртки обобщённой функции.
\begin{Def}
  Пусть $f,g\in D'(\R^n)$. Говорим, что существует свёртка $f\star g$, если $\forall\ \phi\in D(\R^n),\ \forall$ компактного исчерпания единицы $\nu_k\in D(\R^{2n})$ существует предел
\[
  (f\star g,\phi) = \yo k\infty\big( f(x)g(y),\eta_k(x,y)\phi(x+y)\big).
\]
\end{Def}

Почему это хорошо? Почему, если $f,g\in L_1(\R^n)$, то получим то же самое? Какое выражение получится:
\[\Gint{\R^n}\Gint{\R^n} f(\xi)g(y)\eta_k(\xi,y)\phi(\xi+y)\,dy\,d\xi.\]
На любом компакте к какого-то номера последовательность стабилизируется. Поэтому можем применить теорему об ограниченной сходимости (это вы тоже скоро узнаете: все функции ограничены сверху, почти всюду функции сходятся к некоей, тогда интеграл последовательности будет сходиться к интегралу предела). Получится, что в пределе нужный интеграл и будет.

Так мы видим, что определение подходит. Оно не даёт ничего нового для $L_1$.

Есть вопросы к самому определению. Зависит ли предел от выбора компактного исчерпания?
\begin{Task}
  Покажите, что предел не зависит от компактного исчерпания.
\end{Task}
Это утверждение тривиально. По сути оно содержит в себе корректность определение свёртки. Сделаю вам одну подсказку. Есть два компактных исчерпания единицы $\eta_k,\lambda_k$. Перемешаю: $\eta_1,\lambda_1,\eta_2,\lambda_2,\dots$

Есть вопрос ешё вот какой: почему в результате получится обобщённая функция? Предел есть, функция получается линейная относительно $\phi$. Вопрос с непрерывностью. Если последовательность обобщённой функции сходится слабо, то предел "--- обобщённая функция. Мы это сейчас сформулируем без доказательства (можно прочитать у Шилова, оно громоздкое).
\begin{Def}
  Говорят, что последовательность обобщённых функций $f_k\in D'(\Omega)$ слабо сходится к $f\colon D(\Omega)\to \C$, если 
  \[
     \forall\ \phi\in D(\Omega)\pau (f_k,\phi)\to (f,\phi).
  \]
\end{Def}
Можно придумать и другие сходимости, но естественной является именно слабая.

\begin{The}[о полноте $D(\Omega)$ относительно слабой сходимости]
  Пусть $f_k\in D'(\Omega)$ слабо сходится к $f\colon F(\Omega)\to \C$ при $k\to\infty$. Тогда $f\in D'(\Omega)$.
\end{The}
Доказывать не будем. Доказательво, к сожалению, достаточно громозкое. В Шилове доказательство, обратите вниманиме, страницах наверное на пяти. Меня извиняет, что Владимиров тоже так делает: даёт эту теорему без доказательства.
\begin{Task}
  Покажите, что $\phi\mapsto \big(f(x)g(y), \nu_k(x,y)\phi(x+y)\big)$ является линейным непрерывным функционалом в $D(\R^n)$.
\end{Task}
Решение писать не буду, потому что оно тривиально. Хотя первое тоже было тривиально.

Прямое произведение всегда существует, а вот свёртка существует на для всех пар функций. Можно даже пару из $L_1^{loc}(\R^n)$ (суммируемые на каждом компакте) взять. Единицу с единицей мы не свернём.

Какие свойства есть у классической свёртки? Коммутативность. Это действительно так. Мы это сформулируем в виде теоремы небольшой.
\begin{The}[коммутативность свёртки]
  Пусть $f,g\in D'(\R^n)\colon$ существует свёртка $f\star g$. Тогда $\exists$ свёртка $g\star f = f\star g$.
\end{The}

\begin{Proof}
  Пусть $\eta_k\in D(\R^{2n})$ "--- некоторое компактное исчерпание. Имеем
  \[ \big(f(x)g(y),\eta_k(x,y)\phi(x+y)\big) = \Big(f(x),\big(g(y),\eta_k(x,y)\phi(x+y)\big)\Big) = \Big(g(y),\big(f(x),\eta_k(x,y)\phi(x+y)\big)\Big).\]
\end{Proof}

\section{20 октября 2014}
\subsection{Свёртка с обобщённой функцией, имеющей компактный носитель}
Теперь такое утверждение будет. Что у нас получится, если одна из компонент свёртки имеет компактный носитель.
\begin{The}
  Пусть $f,g\in D'(\R^n)$, причём $\supp f$ "--- компакт. Тогда всёртка $f\star g$ существует, причём
  \begin{equation}\label{jora1} \forall\ \phi\in D(\R^n)\pau(f\star g,\phi) = \Big(f(x),\big(g(y),\phi(x+y)\big)\Big) = \Big(g(y),\big(f(x),\phi(x+y)\big)\Big).\end{equation}
\end{The}

Обратите внимание, что самая правая часть формулы устроена каким образом: функцию $f$ применяете к $\phi(x+y)$. Если $y$ фиксирована, то $\phi$ уже имеет компактный носитель, тут всё понятно. А то, что получается, является бесконечно гладкой функцией, но почему это будет с компактным носителем "--- вот в чём вопрос. Дальше $g(y)$ применяем к бесконечно гладкой функции, а должны применять к функции с компактным носителем.

Как обобщённую функцию с компакным носителем доопределить на всё $\R^n$? Находим $\sigma\in D(\R^n)\colon \sigma=1$ в окрестности $\supp f$, и $\forall\ \psi\in C^{\infty}(\R^n)\pau (f,\psi):=(f,\sigma \psi)$. Тогда мы распространили $f\colon C^{\infty}(\R^n)\to \C$. Корректно ли? Проверим
\[\forall\ \psi\in D(\R^n)\pau (f,\psi) = (f\sigma,\psi) = (f,\sigma\psi).\]
Если есть две такие $\sigma_1,\sigma_2$, то $\sigma_1-\sigma_2=0$ в окрестности носителя $f$. Значит, от выбора $\sigma$ выражение не зависит.

Тогда почему в \eqref{jora1} проблем нет? Потому что у $f$ компактный носитель и $g(y)$ применяется к функции с компактным носителем.
\begin{Proof}
  Рассмотрим компактное исчерпание единицы $\eta_k(x,y)\in D(\R^{2n})$, $k=1,2,\dots$ Имеем
\[\big(f(x)g(y),\eta_k(x,y)\phi(x+y)\big).\]
Вот если у неё предел есть, то свёртка существует. Предела нет "--- не существует. Напишем вот в таком силе
\[ = \Big(f(x),\big(g(y),\eta_k(x,y)\phi(x+y)\big)\Big)\]
Вспомним, что $f$ с компактным носителем, а $\sigma$ равна единице в окрестности носителя $f$
\[ = \Big(f(x)\sigma(x),\big(g(y),\eta_k(x,y)\phi(x+y)\big)\Big) = \]
По определению умножения обобщённой функции на бесконечно гладкую
\[=\Big(f(x),\sigma(x)\big(g(y),\eta_k(x,y)\phi(x+y)\big)\Big) = \]
А дальше по линейности функционала $g(y)$.
\[=\Big(f(x),\Big(g(y)\eta_k(x,y)\underbrace{\sigma(x)\phi(x+y)}_{\in D(\R^{2n})}\big)\Big)\]
Если $x$ большое, то $\sigma$ обнулится, если $y$ большое, но $x$ небольшое, то $x+y$ большое, тогда $\phi$ обнулится. И всё это равняется вот такому вот выражению
\[ = \Big(f(x),\big(g(y),\sigma(x)\phi(x+y)\big)\Big),\]
если $k$ такое, что $\eta_k\big|_{\supp\sigma(x)\phi(x+y)} = 1$. Ввиду того, что $\eta_k$ "--- компактное исчерпание единицы, для любого компакта нужное $k$ найдётся. Предел существует просто потому, что стабилизировалась последовательность.

Таким образом,
\begin{multline*}
  \exists\ k_0\colon \forall\ k\ge k_0\pau \big(f(x)g(y),\eta_k(x,y)\phi(x+y)\big)=\Big(f(x),\big(g(y),\sigma(x)\phi(x+y)\big)\Big) = \\
  =\Big(\underbrace{f(x)\sigma(x)}_{f(x)},\big(g(y),\phi(x+y)\big)\Big) = \Big(f(x),\big(g(y),\phi(x+y)\big)\Big),
\end{multline*}
поэтому $\yo k\infty \big(f(x)g(y),\eta_k(x,y)\phi(x+y)\big) = \Big(f(x),\big(g(y),\phi(x+y)\big)\Big)$.

А теперь надо показать, что второе равенство тоже имеет местро. Ну понятно почему: очевидно, что 
\begin{multline*}
  \Big(f(x),\big(g(y),\phi(x+y)\big)\Big) = \Big(f(x)\sigma(x),\big(g(y),\phi(x+y)\big)\Big)=\Big(f(x),\big(g(y),\sigma(x)\phi(x+y)\big)\Big)\overset{\ref{kompramprod}}=\\
  =\Big(g(y),\big(f(x)\sigma(x),\phi(x+y)\big)\Big) = \Big(g(y),\big(f(x),\phi(x+y)\big)\Big).
\end{multline*}
\end{Proof}

В качестве примера мы возьмём $\delta$-функцию. У неё компактный носитель множество из одной точки. Пусть $f\in D'(\R^n)$. Тогда (напишем сперва ответ)
\[ f(x)\star\delta(x) = f(x).\]
В самом деле, $\forall\ \phi\in D(\R^n)$ имеем
\[
  \big(f(x)\star\delta(x),\phi(x)\big) = \Big(f(x),\big(\delta(y),\phi(x+y)\big)\Big) = 
  \big(f(x),\phi(x)\big).
\]

Когда у обобщённой функции носитель компакт, её можно применять к любой бесконечно гладкой, не обязательно имеющей компактный носитель.
\subsection{Дифференция свёртки}
\begin{The}
  Пусть $f,g\in D'(\R^n)$, такие, что существует свёртки $f\star g$. Тогда существуют свёртки $\CP f{x_i}\star g,\ f\star\CP g{x_i}$ и при этом
  \[
    \CP{}{x_i}(f\star g) = \CP{f}{x_i}\star g = f\star\CP g{x_i}.
  \]
\end{The}
Есть такие студенты, которые скажут: вы тут описались, надо было плюс поставить. Нет, не описался.

Обратное, очевидно, неверно: не факт, что если есть свёртки  $\CP f{x_i}\star g,\ \star\CP g{x_i}$, то существует $f\star g$. Простой пример: функции из $L^1_{loc}$.
\begin{Proof}
  Для начала напишу определение производной. Для любой $\phi\in D(\R^n)$, для любого компактного исчерпания единицы $\eta_k\in D(\R^{2n})$, $k=1,2,\dots$
  \[
     \left(\CP{}{x_i}(f\star g),\phi\right) = - \left(f\star g,\CP \phi{x_i}\right) =
    -\yo k\infty\left(f(x)g(y),\eta_k(x,y)\CP{}{x_i}\phi(x+y)\right).
  \]
  Это с одной стороны. А теперь другую сторону равенства хотим переписать в таком же виде. Хотим показать, что существует предел
  \[
    \left(\CP f{x_i}\star g,\phi\right) = \yo k\infty \left( \CP f{x_i}(x)g(y),\eta_k(x,y)\phi(x+y)\right),
  \]
  причём
  \[
    -\yo k\infty\left(f(x)g(y),\eta_k(x,y)\CP{}{x_i}\phi(x+y)\right) = \yo k\infty \left( \CP f{x_i}(x)g(y),\eta_k(x,y)\phi(x+y)\right).
  \]
  Вот для этого разберёмся, в чём разница в выражениях под знаком предела. В самом деле
  \begin{multline*}
    \left(\CP f{x_i}(x)g(y),\eta_k(x,y)\phi(x+y)\right) \overset{\ref{DefOtimes}} = 
    \left(\CP f{x_i}(x),\big(g(y),\eta_k(x,y)\phi(x+y)\big)\right)=\\
   -\left(f(x),\CP{}{x_i}\big(g(y),\eta_k(x,y)\phi(x+y)\big)\right)=
   -\left(f(x),\Big(g(y),\CP{}{x_i}\big(\eta_k(x,y)\phi(x+y)\big)\Big)\right)=\\
   -\left(f(x),\Big(g(y),\CP{\eta_k(x,y)}{x_i}\phi(x+y)\Big)\right)
   -\left(f(x),\Big(g(y),\eta_k(x,y)\CP{\phi(x+y)}{x_i}\Big)\right). % (Я)
  \end{multline*}
  Покажем, что предел от первого слагаемого есть ноль, то есть
  \begin{equation} \label{Yi}
    \yo k\infty\left(f(x),\Big(g(y),\CP{\eta_k(x,y)}{x_i}\phi(x+y)\Big)\right).
  \end{equation}
  Догадаться, чем воспользоваться очень непросто. Обычно начинают какие-то оценки писать, и ничего не получается. Нужно построить новое компактное исчерпание единицы. Положим $\lambda_k(x,y) = \eta_k(x,y) + \CP{\eta_k}{x_i}(x,y)$, $k=1,2,\dots$ Очевидно, что $\lambda_k(x,y)$ также будет компактным исчерпанием единицы. Предел о определении свёртки не зависит от выбора компактного исчерпания единицы
  \[
    \yo k\infty\Big(f(x),\big(g(y),\lambda_k(x,y)\phi(x+y)\big)\Big) = \yo_k\infty\Big(f(x),\big(g(y),\eta_k(x,y)\phi(x+y)\big)\Big).
  \]
  Однако
  \[
    \Big(f(x),\big(g(y),\lambda_k(x,y)\phi(x+y)\big)\Big) = \Big(f(x),\big(g(y),\eta_k(x,y)\phi(x+y)\big)\Big) +\left(f(x),\Big(g(y),\CP{\eta_k(x,y)}{x_i}\phi(x+y)\Big)\right).
  \]
  Откуда следует \eqref{Yi}.

  Ввиду (Я) соотношение (Ы) влечёт за собой (ЯЯ). Соотношение (ЯЯ) в свою очередь означает, что
\begin{multline*}
  \left( \CP{}{x_i}(f\star g),\phi\right) = -\left(f\star g,\CP\phi{x_i}\right) = 
 -\yo k\infty \left(f(x)g(y),\eta_k(x,y)\CP\phi{x_i}(x+y)\right) =\\
  \yo k\infty \left(\CP f{x_i}(x)g(y),\eta_k(x,y)\phi(x+y)\right) = \left(\CP f{x_i}\star g,\phi\right).
\end{multline*}
Другими словами, свёртка $\CP f{x_i}\star g$ существует и при этом
 \[
  \CP{f\star g}{x_i} = \CP f{x_i}\star g.
 \]
Последнее ввиду коммутативности свёртки позволяет утверждать, что существует также $f\star \CP g{x_i}$, причём $\CP{(f\star g)}{x_i} = f\star\CP g{x_i}$. Что и требовалось доказать.
\end{Proof}

\subsection{Теорема существования и теорема единственности}
Пусть $\mathcal L = \sum\limits_{|a|\le m}a_\alpha\dl^\alpha$, $a_\alpha\in \C$ "--- дифференциальный оператор с постоянными коэффициентами, $\alpha \hm= \ve \alpha n$ "--- мультииндекс, $|\alpha| = \alpha_1+\ldots+\alpha_n$, $\dl^\alpha =\CP{^{|\alpha|}}{x^{\alpha_1}\dots\dl x^{\alpha^n}}$.

\begin{Def}
  $\mathcal E(x)\in D'(\R^n)$ называется фундаменальным решением оператора $\mathcal L$, если $\mathcal L \mathcal E = \delta(x)$.
\end{Def}

В качестве примера возьмём $\mathcal L = \frac{d}{dx}$. Тогда $\mathcal E(x) = \theta(x) = \begin{cases}1,&x>0,\\0,&x<0\end{cases}$ "--- фундаментальное решение $\mathcal L$. В самом деле, имеем, $\theta'(x) = \delta(x)$.

У нас осталось пять минут. Ровно столько, сколько нужно, чтобы доказать две теоремы.
\begin{The}
  Пусть $f(x)\in D'(\R^n)$, такая, что существует свёртка $u = f\star \mathcal E$, где $\mathcal E(x)$ "--- фундаментальное решение оператора $\mathcal L$. Тогда $\mathcal L u = f$.
\end{The}
\begin{Proof}
  Имеем $\mathcal L u = \mathcal L(f\star \mathcal E) = f\star \mathcal L \mathcal E = f\star \delta = f$. Ведь у свёртки можно дифференцировать только одну компоненту.
\end{Proof}

\begin{The}
  Пусть $u\in D'(\R^n)$ "--- решение уравнения $\mathcal L u = f$, такое, что существует свёртка $u\star \mathcal E$, где $\mathcal E$ "--- фундаментальное решение оператора $\mathcal L$. Тогда $u = f\star \mathcal E$.
\end{The}

\begin{Proof}
  Имеем $\mathcal L(u\star\mathcal E) = u\star \mathcal L\mathcal E = u\star \delta = u$. С другой стороны
\[
  \mathcal L(u\star \mathcal E) = \mathcal L u\star\mathcal E = f\star\mathcal E\imp u = f\star \mathcal E.
 \]
Я на две минуты вас обманул всего.
\end{Proof}
