\section{2 Марта}
Чтобы обобщённое решение стало классическим требуется некоторое дополнительное условие.
\begin{The}
  Пусть $g\in C^2\big(\R^n\times[0,\infty)\big)$, все производные $\dl^\a f\in\M$, $|\a|\le 2$, $\a = (\a_1,\dots,\a_n,\a_{n+1})$ "--- мультииндекс, $u_0\in C(\R^n)\cap\L_\infty(\R^n)$. Тогда задача Коши
\begin{equation}\label{LapKo}
\begin{cases}
u_t = a^2\,\Delta u + f(x,t),&x\in\R^n,\ t>0,\\
u(x,0) = u_0(x)
\end{cases}
\end{equation}
имеет, причём единственное, решение $u\in C^2\big(\R^n\times(0,\infty)\big)\cap\big(\R^n\times[0,\infty)\big)$ (так называемое классическое решение). Это решение выражается формулой Пуассона (мы уже писали, но сейчас будет некоторый нюанс)
\begin{equation}\label{LapKo2}
  u(x,t) = 
    \underbrace{
\int\limits_0^t\Gint{\R^n}\frac{f(\xi,\tau)}{\left(2\,a\,\sqrt{\pi(t-\tau)}\right)^n}\,e^{-\frac{|x-\xi|^2}{4\,a^2\,(t-\tau)}}\,d\xi\,d\tau
}_{I_1(x,t)} + 
    \underbrace{\frac1{\left(2\,a\,\sqrt{\pi\,t}\right)^n}\Gint{\R^n}\,e^{-\frac{|x-\xi|^2}{4\,a^2\,t}}\,d\xi}_{I_2(x,t)}.
\end{equation}
\end{The}
\begin{Proof}
Согласно предыдущей теореме, задача Коши \eqref{LapKo} имеет единственное обобщённое решение, которое выражается формулой Пуассона для обобщённого решения, совпадающей с \eqref{LapKo2} при $t>0$. Нужно доказать, что получаемое решение имеет нужный класс гладкости.

Пусть $\Til u$ "--- обобщённое решение задачи Коши для уравнения теплопроводности. При этом $\Til u$ задаётся формулой
\[
  \Til u(x,t) =\theta(t)
    \int\limits_0^t\Gint{\R^n}\frac{f(\xi,\tau)}{\left(2\,a\,\sqrt{\pi(t-\tau)}\right)^n}\,e^{-\frac{|x-\xi|^2}{4\,a^2\,(t-\tau)}}\,d\xi\,d\tau + 
    \frac1{\left(2\,a\,\sqrt{\pi\,t}\right)^n}\Gint{\R^n}\,e^{-\frac{|x-\xi|^2}{4\,a^2\,t}}\,d\xi.
\]
Если все нужные производные обобщённого решение можно посчитать в классическом смысле, то это всё равно что классическое решение обобщённого решение можно посчитать в классическом смысле, то это всё равно что классическое решение. Функция $\Til u$ удовлетворяет уравнению
\[
  \Til u_t = a^2\,\Delta\Til u + \Til f(x,t) + u_0(x)\,\delta(t).
\]
Покажем, что правая часть \eqref{LapKo2} принадлежит классу $C^2\big(\R^n\times(0,\infty)\big)\cap C(\R^n\times[0,\infty)\big)$. Покажем сначала, что $I_1\in C\big(\R^n\times[0,\infty)\big)$. Пусть $(x,t)\in\R^n\times[0,\infty)$. Покажем что $I_1$ непрерывно в точке $(x,t)$. Имеем
\[
  \big|I_1(x,t) - I_1(x_0,t_0)\big| =
  \bigg|
    \int\limits_0^t
    \Gint{\R^n}\frac{f(\xi,\tau)}{\left(2\,a\,\sqrt{\pi(t-\tau)}\right)^n} e^{-\frac{|x-\xi|^2}{4\,a^2\,(t-\tau)}}\,d\xi\,d\tau
    -\int\limits_0^{t_0}
    \Gint{\R^n}\frac{f(\xi,\tau)}{\left(2\,a\,\sqrt{\pi(t-\tau)}\right)^n} e^{-\frac{|x_0-\xi|^2}{4\,a^2\,(t_0-\tau)}}\,d\xi\,d\tau.
\]
Имеем
\begin{multline*}
 I_1(x,t) = 
\int\limits_0^t\Gint{\R^n}\frac{f(\xi,\tau)}{\left(2\,a\,\sqrt{\pi(t-\tau)}\right)^n}\,e^{-\frac{|x-\xi|^2}{4\,a^2\,(t-\tau)}}\,d\xi\,d\tau
= \left\{
  \begin{matrix}
  y = \frac{x-\xi}{2\,a\,\sqrt{t-\tau}}\\
  \xi = x-2\,a\,\sqrt{t-\tau}\,y\\
  d\xi = \left(2\,a\,\sqrt{t-\tau}\right)^n\,dy
  \end{matrix}\right\}=\\=
  \frac1{\pi^{n/1}}\int\limits_0^t\Gint{\R^n} f(x-2\,a\,sqrt{t-\tau}y,\tau)e^{-|y|^2}\,dy\,\tau = 
  \left\{\begin{matrix}
    z = t-\tau\\
    d\tau = -dz
\end{matrix}\right\} =
   \frac1{\pi^{n/2}}\int\limits_0^t \Gint{\R^n}f(x-2\,a\,\sqrt z\,y,t-z)e^{-|y|^2}\,dy\,dz.
\end{multline*}
Вернёмся к оценке разности
\[
   \big|I_1(x,t)-I_1(x_0,t_0)\big| = 
   \frac1{\pi^{n/2}}\bigg|
   \int\limits_0^t \Gint{\R^n}f(x-2\,a\,\sqrt z\,y,t-z)e^{-|y|^2}\,dy\,dz - 
   \int\limits_0^{t_0} \Gint{\R^n}f(x-2\,a\,\sqrt z\,y,t_0-z)e^{-|y|^2}\,dy\,dz.
\]
  Есть загвоздка, мы интегрируем по $\R^n$, надо посмотреть на экспоненту. Ещё один момент, у нас функция класса $\M$. Разобьём интеграл на два. Сначала будет интегрировать по замкнутому шару, а потом до дополнению (последний интеграл будет маленький, а на компакте мы воспользуемся равномерной непрерывностью). 
\begin{multline*}
\int\limits_0^t \Gint{\R^n}f(x-2a\sqrt z\,y,t-z)e^{-|y|^2}\,dy\,dz -
\int\limits_0^{t_0} \Gint{\R^n}f(x_0-2a\sqrt z\,y,t_0-z)e^{-|y|^2}\,dy\,dz=\\
=
\int\limits_{t_0}^t \Gint{\R^n}f(x-2a\sqrt z\,y,t-z)e^{-|y|^2}\,dy\,dz +
\int\limits_{0}^{t_0} \Gint{\R^n}
  \big(f(x-2a\sqrt z\,y,t-z)-f(x_0-2a\sqrt z\,y,t_0-z)\big)e^{-|y|^2}\,dy\,dz.
\end{multline*}

Оценим для некоторого $\e\in(0,1)$ и $|t-t_0|<\e$
\[
  \bigg|\int\limits_{t_0}^t\Gint{\R^n}f(x-2a\sqrt zy,t-z)e^{-|y|^2}\,dy\,dz\bigg|\le\underbrace{|t-t_0|}_{<\e}\|f\|_{\L_{\infty}(\R^n\times[0,t_0+1])}\Gint{\R^n}e^{-|y|^2}\,dy<\infty\
\]

Тогда
\begin{multline*}
  \int\limits_0^{t_0}\big|\underbrace{f(x-2a\sqrt z y,t-z)-f(x_0-2a\sqrt z y,t_0-z)}_{g(x,x_0,t,t_0,y,z)}\big|e^{-|y|^2}\,dy\,dz = \\ =
  \int\limits_0^{t_0}\Gint{B_r}\big|g(x,x_0,t,t_0,y,z)\big|e^{-|y|^2}\,dy\,dz + 
  \int\limits_0^{t_0}\Gint{\R^n\dd B_r}\big|g(x,x_0,t,t_0,y,z)\big|e^{-|y|^2}\,dy\,dz
\end{multline*}
Первый интеграл по равномерной непрерывности, а второй так же, как первое слагаемое, ведь $f$ из $\M$
  \[\int\limits_0^{t_0}\Gint{\R^n\dd B_r}\big|g(x,x_0,t,t_0,y,z)\big|e^{-|y|^2}\,dy\,dz\le
  2\,t_0\,\|f\|_{\L_\infty\left(\R^n\times[0,t_0]\right)}\Gint{\R^n\dd B_r}e^{-|y|^2}\,dy
\]
А другой интеграл
\[
  \int\limits_0^{t_0}\Gint{B_r}\big|g(x,x_0,t,t_0,y,z)\big|e^{-|y|^2}\,dy\,dz\le
 t_0\sup\limits_{\substack{|x-x_0|<\e\\|t-t_0|<\e\\y\in \ol B_r\\ z\in[0,t_0]}} \big|g(x,x_0,t,t_0,y,z)\big|
\]
Зафиксировав $r$ выбираем $\e$. А потом $r\to\infty$.

Непрерывность $I_1$ мы показали. Покажем непрерывность второго слагаемого, пользуясь фактом $u_0\in \L_\infty(\R^n)\cap C(\R^n)$
\begin{multline*}
 I_2(x,t) = \frac1{\left(2\,a\,\sqrt{\pi\,t}\right)^n}\Gint{\R^n}u_0(\xi)\,e^{-\frac{|x-\xi|^2}{4\,a^2\,t}}\,d\xi = \\
\cmt{Тем же самым способом воспользуемся. Выполним замену}\\
=\left\{\begin{aligned}
y&=\frac{x-\xi}{2\,a\,\sqrt{t}}\\
\xi&=x-2\,a\,\sqrt t\,y\\
d\xi &= (2\,a\,\sqrt t)^n\,dy
\end{aligned}\right\}=
  \frac1{\pi^{n/2}}\Gint{\R^n} u_0(x-2\,a\,\sqrt t\,y)e^{-y^2}\,dy.
\end{multline*}
Повторяя предыдущие рассуждения, получим, что $I_2\in\C\big(\R^n\times[0,\infty)\big)$.

А ещё надо доказать, что $I_1,I_2\in\M$. Мы можем написать соответствующую оценку. Класс $\M$ означает ограниченность в каждой конечной полосе. У нас есть суммируемость экспоненты и условия на $u_0$. Например, в случае $I_1(x,t)$
\begin{multline*}
I_1(x,t) = \int\limits_0^t\Gint{\R^n}\frac{f(\xi,\tau)}{\left(2\,a\,\sqrt{\pi(t-\tau)}\right)^n}\,e^{-\frac{|x-\xi|^2}{4\,a^2\,(t-\tau)}}\,d\xi\,d\tau = 
\left\{\begin{aligned}
y&=\frac{x-\xi}{2\,a\,\sqrt{t-\tau}}\\
\xi &=x-2\,a\,\sqrt{t-\tau}\,y\\
d\xi&=(2\,a\,\sqrt{t-\tau})^n\,dy
\end{aligned}\right\}=\\
=\frac1{\pi^{n/2}}\int\limits_0^t\Gint{\R^n} f\underbrace{(x-2\,a\,\sqrt{t-\tau}\,y,\tau)}_{\in\M,\text{ т.\,е. }\forall\ T>0\pau f\in\L_\infty\big(\R^n\times[0,T]\big)}e^{-|y|^2}\,dy\,d\tau\le\\\le
\frac1{\pi^{n/2}}\|f\|_{\L_\infty\big(\R^n\times[0,\infty]\big)}\int\limits_0^t\Gint{\R^n}e^{-|y|^2}\,dy\,d\tau = t\|f\|_{\L_\infty\big(\R^n\times[0,t]\big)}<\infty.
\end{multline*}
Аналогично имеем
\[
  I_2(x,t) = \frac1{\pi^{n/2}}\Gint{\R^n}u_0(x-2\,a\,\sqrt t\,y)e^{-|y|^2}\,dy\le
\frac1{\pi^{n/2}}\|u_0\|_{\L_\infty(\R^n)}\Gint{\R^n} e^{-|y|^2}\,dy = \|u_0\|_{\L_\infty(\R^n)}.
\]

Надо показать теперь дифференцируемость, а именно покажем, что $I_1\in C^2\big(\R^n\times(0,\infty)\big)$. Мы делали замену времени $s = t-\tau$, переставляли пределы интегрирования.
\[
 I_1(x,t) = \frac1{\pi^{n/2}}\int\limits_0^t\Gint{\R^n}f(x-2\,a\,\sqrt s\,y,t-s)e^{-|y|^2}\,dy\,ds.
\]
Отсюда видно, что
\[
  \dl^\a_x I_1(x,t) = \frac1{\pi^{n/2}}\int\limits_0^t\Gint{\R^n}\dl^\a_x f(x-2\,a\,\sqrt s\,y,t-s)e^{-|y|^2}\,dy\,ds,
\]
где $\dl^\a_x = \CP{^{|\a|}}{x_1^{\a_1}\dots x_n^{\a_n}}$, $\a = (\a_1,\dots,\a_n)$, $|\a|=\a_1+\dots+\a_n$ и $|\a|\le2$. Предполагается, что вы умеете дифференцировать интеграл с параметром. С существованием производной здесь всё в порядке. Все производные ограничены на ограниченном по времени промежутке. Тогда за счёт экспоненты подынтегральное выражение является суммируемым. Я докажу вам теорему, по которой можно будет дифференцировать этот интеграл по $x$. А по времени получается следущее
\[
\dl_t I_1(x,t) =
 \frac1{\pi^{n/2}}\Gint{\R^n} \underbrace{f(x-2\,a\,\sqrt s\,y,t-s)}_{\in\M}\,e^{-|y|^2}\,dy + 
  \frac1{\pi^{n/2}}\int\limits_0^t\Gint{\R^n}\dl_t\,\underbrace{f(x-2\,a\,\sqrt s\,y,t-s)}_{\in\M}
    \,e^{-|y|^2}\,dy\,ds.
\]
Далее вторую
\[
\dl^2_t I_1(x,t) = 
  \frac2{\pi^{n/2}}\Gint{\R^n}\dl_t\,\underbrace{f(x-2\,a\,sqrt s\,y,t-s)}_{\in\M}\,e^{-|y|^2}\,dy +
  \frac1{\pi^{n/2}}\int\limits_0^t\Gint{\R^n}\dl^2_t
  \underbrace{ f(x-2\,a\,\sqrt s\,y\,t-s)}_{\in\M}\,e^{-|y|^2}\,dy\,ds.
\]
Нам здесь нужна гладкость подынтегральной функции порядка два в замыкании области. Иначе при $t=s$ были бы проблемы.

А смешанная производная
\[
\dl_x\dl_t I_1(x,t) = \frac1{\pi^{n/2}}\Gint{\R^n}\dl_x f(x-2\,a\,\sqrt s\,y,t-s)\,e^{-|y|^2}\,dy + 
  \frac1{\pi^{n/2}}\int\limits_0^t\Gint{\R^n}\dl_x\dl_t f(x-2\,a\,\sqrt s\,y,t-a)\,e^{-|y|^2}\,dy\,ds.
\]
Все подынтегральные множители экспоненты из класса $\M$.

Таким образом, $I_1\in C^2\big(\R^n\times[0,\infty)\big)$. Аналогично $I_2$.
\[
  \dl_{x,t}^\b],I_2(x,t) = \sum\limits_{\substack{\a+\g+\nu=\b\\\a,\g,\nu\in\Z_+}}
  \dl^\nu\frac1{(2\,a\,\sqrt{\pi\,t})^n}
  \Gint{\R^n}
   u_0(\xi)\dl_x^\a \dl_t^\g 
   e^{-\frac{|x-\xi|^2}{4\,a^2\,t}}
  \,d\xi.
\]
Здесь $\b\le2$. При дифференцировании 
$  \dl^\nu\frac1{(2\,a\,\sqrt{\pi\,t})^n}$ растёт степень знаменателя. А под интегралом экспонента всё забивает. Как функция от $\xi$
\[
   \forall\ t>0,\ \forall\ x\in\R^n\pau 
   u_0(\xi)\dl_x^\a \dl_t^\g 
   e^{-\frac{|x-\xi|^2}{4\,a^2\,t}}
  \in\L_1(\R^n).
\]
Видим, что $I_2$ на самом деле бесконечно можно дифференцировать, ведь экспонента всё забивает, любую степень. Но это всё так, пока есть требование гладкости функций из условия до границы.

Тем самым, $I_2\in C^\infty\big(\R^n\times(0,\infty)\big)$. Поэтому $I_1+I_2\in\C^2\big(\R^n\times(0,\infty)\big)\cap C\big(\R^n\times[0,\infty)\big)\cap \M$.

Таким образом, $u(x,t) = I_1(x,t) + I_2(x,t)$ является классическим решением задачи Коши для уравнения теплопроводности. Его единственность вытекает из теоремы единственности обобщённого решения и следующего упражнения.
\end{Proof}

\begin{Task}
  Пусть $v,w\in\D'(\R^n)$, причём $v(x)\delta(t) = w(x)\delta(t)$. Тогда $v=w$.
\end{Task}
\begin{Solution}
Возьмём $\phi\in\D(\R^n)$, $\eta\in\D(\R)$. Имеем
\[
  \big(v(x)\delta(t),\phi(x)\eta(t)\big) = 
  \big(w(x)\delta(t),\phi(x)\eta(t)\big).
\]
Осюда по определению
\[
  \big(v(x),\phi(x)\big)\eta(0) = \big(w(x),\phi(x)\big)\eta(0).
\]
Возьмём $\eta\colon \eta(0)\ne0$. Осюда следует, что $\big(v(x),\phi(x)\big) = \big(w(x),\phi(x)\big)$. Значит, $v=w$.
\end{Solution}
