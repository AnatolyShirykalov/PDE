\section{2 Марта}
Чтобы обобщённое решение стало классическим требуется некоторое дополнительное условие.
\begin{The}
  Пусть $g\in C^2\big(\R^n\times[0,\infty)\big)$, все производные $\dl^\a f\in\M$, $|\a|\le 2$, $\a = (\a_1,\dots,\a_n,\a_{n+1})$ "--- мультииндекс, $u_0\in C(\R^n)\cap\L_\infty(\R^n)$. Тогда задача Коши
\begin{equation}\label{LapKo}
\begin{cases}
u_t = a^2\,\Delta u + f(x,t),&x\in\R^n,\ t>0,\\
u(x,0) = u_0(x)
\end{cases}
\end{equation}
имеет, причём единственное, решение $u\in C^2\big(\R^n\times(0,\infty)\big)\cap\big(\R^n\times[0,\infty)\big)$ (так называемое классическое решение). Это решение выражается формулой Пуассона (мы уже писали, но сейчас будет некоторый нюанс)
\begin{equation}\label{LapKo2}
  u(x,t) = 
    \underbrace{
\int\limits_0^t\Gint{\R^n}\frac{f(\xi,\tau)}{\left(2\,a\,\sqrt{\pi(t-\tau)}\right)^n}\,e^{-\frac{|x-\xi|^2}{4\,a^2\,(t-\tau)}}\,d\xi\,d\tau
}_{I_1(x,t)} + 
    \underbrace{\frac1{\left(2\,a\,\sqrt{\pi\,t}\right)^n}\Gint{\R^n}\,e^{-\frac{|x-\xi|^2}{4\,a^2\,t}}\,d\xi}_{I_2(x,t)}.
\end{equation}
\end{The}
\begin{Proof}
Согласно предыдущей теореме, задача Коши \eqref{LapKo} имеет единственное обобщённое решение, которое выражается формулой Пуассона для обобщённого решения, совпадающей с \eqref{LapKo2} при $t>0$. Нужно доказать, что получаемое решение имеет нужный класс гладкости.

Пусть $\Til u$ "--- обобщённое решение задачи Коши для уравнения теплопроводности. При этом $\Til u$ задаётся формулой
\[
  \Til u(x,t) =\theta(t)
    \int\limits_0^t\Gint{\R^n}\frac{f(\xi,\tau)}{\left(2\,a\,\sqrt{\pi(t-\tau)}\right)^n}\,e^{-\frac{|x-\xi|^2}{4\,a^2\,(t-\tau)}}\,d\xi\,d\tau + 
    \frac1{\left(2\,a\,\sqrt{\pi\,t}\right)^n}\Gint{\R^n}\,e^{-\frac{|x-\xi|^2}{4\,a^2\,t}}\,d\xi.
\]
Если все нужные производные обобщённого решение можно посчитать в классическом смысле, то это всё равно что классическое решение обобщённого решение можно посчитать в классическом смысле, то это всё равно что классическое решение. Функция $\Til u$ удовлетворяет уравнению
\[
  \Til u_t = a^2\,\Delta\Til u + \Til f(x,t) + u_0(x)\,\delta(t).
\]
Покажем, что правая часть \eqref{LapKo2} принадлежит классу $C^2\big(\R^n\times(0,\infty)\big)\cap C(\R^n\times[0,\infty)\big)$. Покажем сначала, что $I_1\in C\big(\R^n\times[0,\infty)\big)$. Пусть $(x,t)\in\R^n\times[0,\infty)$. Покажем что $I_1$ непрерывно в точке $(x,t)$. Имеем
\[
  \big|I_1(x,t) - I_1(x_0,t_0)\big| =
  \bigg|
    \int\limits_0^t
    \Gint{\R^n}\frac{f(\xi,\tau)}{\left(2\,a\,\sqrt{\pi(t-\tau)}\right)^n} e^{-\frac{|x-\xi|^2}{4\,a^2\,(t-\tau)}}\,d\xi\,d\tau
    -\int\limits_0^{t_0}
    \Gint{\R^n}\frac{f(\xi,\tau)}{\left(2\,a\,\sqrt{\pi(t-\tau)}\right)^n} e^{-\frac{|x_0-\xi|^2}{4\,a^2\,(t_0-\tau)}}\,d\xi\,d\tau.
\]
Имеем
\begin{multline*}
 I_1(x,t) = 
\int\limits_0^t\Gint{\R^n}\frac{f(\xi,\tau)}{\left(2\,a\,\sqrt{\pi(t-\tau)}\right)^n}\,e^{-\frac{|x-\xi|^2}{4\,a^2\,(t-\tau)}}\,d\xi\,d\tau
= \left\{
  \begin{matrix}
  y = \frac{x-\xi}{2\,a\,\sqrt{t-\tau}}\\
  \xi = x-2\,a\,\sqrt{t-\tau}\,y\\
  d\xi = \left(2\,a\,\sqrt{t-\tau}\right)^n\,dy
  \end{matrix}\right\}=\\=
  \frac1{\pi^{n/1}}\int\limits_0^t\Gint{\R^n} f(x-2\,a\,sqrt{t-\tau}y,\tau)e^{-|y|^2}\,dy\,\tau = 
  \left\{\begin{matrix}
    z = t-\tau\\
    d\tau = -dz
\end{matrix}\right\} =
   \frac1{\pi^{n/2}}\int\limits_0^t \Gint{\R^n}f(x-2\,a\,\sqrt z\,y,t-z)e^{-|y|^2}\,dy\,dz.
\end{multline*}
Вернёмся к оценке разности
\[
   \big|I_1(x,t)-I_1(x_0,t_0)\big| = 
   \frac1{\pi^{n/2}}\bigg|
   \int\limits_0^t \Gint{\R^n}f(x-2\,a\,\sqrt z\,y,t-z)e^{-|y|^2}\,dy\,dz - 
   \int\limits_0^{t_0} \Gint{\R^n}f(x-2\,a\,\sqrt z\,y,t_0-z)e^{-|y|^2}\,dy\,dz.
\]
  Есть загвоздка, мы интегрируем по $\R^n$, надо посмотреть на экспоненту. Ещё один момент, у нас функция класса $\M$. Разобьём интеграл на два. Сначала будет интегрировать по замкнутому шару, а потом до дополнению (последний интеграл будет маленький, а на компакте мы воспользуемся равномерной непрерывностью). 
\begin{multline*}
\int\limits_0^t \Gint{\R^n}f(x-2a\sqrt z\,y,t-z)e^{-|y|^2}\,dy\,dz -
\int\limits_0^{t_0} \Gint{\R^n}f(x_0-2a\sqrt z\,y,t_0-z)e^{-|y|^2}\,dy\,dz=\\
=
\int\limits_{t_0}^t \Gint{\R^n}f(x-2a\sqrt z\,y,t-z)e^{-|y|^2}\,dy\,dz +
\int\limits_{0}^{t_0} \Gint{\R^n}
  \big(f(x-2a\sqrt z\,y,t-z)-f(x_0-2a\sqrt z\,y,t_0-z)\big)e^{-|y|^2}\,dy\,dz.
\end{multline*}

Оценим для некоторого $\e\in(0,1)$ и $|t-t_0|<\e$
\[
  \bigg|\int\limits_{t_0}^t\Gint{\R^n}f(x-2a\sqrt zy,t-z)e^{-|y|^2}\,dy\,dz\bigg|\le\underbrace{|t-t_0|}_{<\e}\|f\|_{\L_{\infty}(\R^n\times[0,t_0+1])}\Gint{\R^n}e^{-|y|^2}\,dy<\infty\
\]

Тогда
\begin{multline*}
  \int\limits_0^{t_0}\big|\underbrace{f(x-2a\sqrt z y,t-z)-f(x_0-2a\sqrt z y,t_0-z)}_{g(x,x_0,t,t_0,y,z)}\big|e^{-|y|^2}\,dy\,dz = \\ =
  \int\limits_0^{t_0}\Gint{B_r}\big|g(x,x_0,t,t_0,y,z)\big|e^{-|y|^2}\,dy\,dz + 
  \int\limits_0^{t_0}\Gint{\R^n\dd B_r}\big|g(x,x_0,t,t_0,y,z)\big|e^{-|y|^2}\,dy\,dz
\end{multline*}
Первый интеграл по равномерной непрерывности, а второй так же, как первое слагаемое, ведь $f$ из $\M$
  \[\int\limits_0^{t_0}\Gint{\R^n\dd B_r}\big|g(x,x_0,t,t_0,y,z)\big|e^{-|y|^2}\,dy\,dz\le
  2\,t_0\,\|f\|_{\L_\infty\left(\R^n\times[0,t_0]\right)}\Gint{\R^n\dd B_r}e^{-|y|^2}\,dy
\]
А другой интеграл
\[
  \int\limits_0^{t_0}\Gint{B_r}\big|g(x,x_0,t,t_0,y,z)\big|e^{-|y|^2}\,dy\,dz\le
 t_0\sup\limits_{\substack{|x-x_0|<\e\\|t-t_0|<\e\\y\in \ol B_r\\ z\in[0,t_0]}} \big|g(x,x_0,t,t_0,y,z)\big|
\]
Зафиксировав $r$ выбираем $\e$. А потом $r\to\infty$.
\end{Proof}

