\section{24 ноября 2014}
\subsection{Фундаментальное решение оператора Лапласа}
Доказали две вспомогательные леммы в прошлый раз, а теперь сформулируем теорему.
\begin{The}
  Пусть 
 \[
  \E_n(x) = \begin{cases}
  -\frac1{(n-2)|S_1|}\cdot\frac1{|x|^{n-2}},&n\ge 3;\\
  \frac1{2\pi}\ln|x|,&n=2.
\end{cases}
  \]
Тогда $\Delta\E_n(x) = \delta(x)$, то есть $\E_n(x)$  является фундаментальным решением оператора Лапласа (оператора $\Delta = \RY i1n\CP{^2}{x_i^2}$).
\end{The}

Оператор хорошо известен и в математике и в приложениях.
\begin{Proof}
  Пусть $\phi\in \D(\R^n)$. Тогда оператор Лапласа, применённый
\[
  \big(\Delta\E_n(x),\phi(x)\big) = \big(\E_n(x),\Delta\phi(x)\big) = 
\]
В случае $n=2$ особенность $\ln|x|$ в нуле суммируема. Получаем там $x\ln |x|$, что в нуле стремится к нулю. В случае $n\ge3$ тоже получается величина, которая при $x\to0$ стремится к нулю. А значит, последнее выражение можно записать в виде интеграла
\[
  = \Gint{\R_n}\E_n(x)\Delta\phi(x)\,dx = 
\]
А дальше хочется оператор Лапласа перебрасывать обратно только уже в классическом смысле. Для этого мы доказали теорему \ref{FROL2}. Самое сложное в теореме "--- это условие $C^1(\W)\cap C(\ol\W)$. $\phi(x)$ "--- хорошая функция, бесконечно гладкая, а $\E_n(x)$ "--- плохая, у неё особенность в нуле. И более того, область $\R^n$ не является ограниченной, а в теореме требуется именно ограниченная. Поэтому мы поступим так.
\[
  = \yo\e{+0}\Gint{B_R\dd \ol B_\e}\E_n(x)\Delta\phi(x)\,dx.
\]
Здесь $R>0$ "--- некоторое вещественное число, такое, что $\supp\phi\subset B_R$. Интеграл у нас хороший. Имеем
\[
  \Gint{B_R\dd \ol B_\e}\E_n(x)\Delta\phi(x)\,dx = \Gint{B_R\dd\ol B_\e}\E_n(x)\RY i1n\CP{^2\phi}{x_i^2}\,dx = 
  \RY i1n\Gint{B_R\dd B_\e}\E_n(x)\CP{^2\phi}{x_i^2}(x)\,dx = 
\]
Дальше применим ту теорему \ref{FROL2}
\[
  \RY i1n\Gint{\dl(B_R\dd B_\e)}\E_n(x)\CP{\phi(x)}{x_i}\cos(\VE\nu,x_i)\,dS
  - \RY i1n\Gint{B_R\dd B_\e}\CP{\E_n(x)}{x_i}\CP{\phi(x)}{x_i}\,dx=
\]
И ещё раз применяем теорему, а заодно делаем полезное преобразование в первом интеграле $\RY i1n\CP{\phi(x)}{x_i}\cos(\VE\nu,x_i) = \CP\phi{\VE\nu}$.
\begin{multline*}
  = \Gint{\dl(B_R\dd B_\e)}\E_n(x) \underbrace{\RY i1n\CP{\phi(x)}{x_i}\cos(\VE\nu,x_i)}_{\CP\phi{\VE\nu}}\,dS - 
    \Gint{\dl(B_R\dd B_\e)} \underbrace{\RY i1n\CP{\phi(x)}{x_i}\cos(\VE\nu,x_i)}_{\CP\phi{\VE\nu}}\phi\,dS + 
    \Gint{B_R\dd B_\e}\Delta\E_n(x)\phi(x)\,dx = \\
  =\Gint{\dl(B_R\dd B_\e)}\E_n(x)\CP\phi{\VE\nu}\,ds - \Gint{\dl(B_R\dd B_\e)}\CP{\E_n(x)}{\VE\nu}\phi\,ds +
    \Gint{B_R\dd B_\e}\Delta\E_n(x)\phi(x)\,dx.
\end{multline*}
Утверждается, что последнее слагаемое равно нулю. Покажем, что $\Delta\E_n(x)=0$ для всех $x\in\R^n\dd\{0\}$. В самом деле, в многомерных полярных координатах оператор Лапласа имеет вида
\[
  \Delta = \CP{^2}{r^2}+\frac{n-1}r+\frac1{r^2}\Delta_{S_1},
\]
где $\Delta_{S_1}$ "--- оператор Лапласа"--~Бельтрами на единичной сфере (например, $\Delta_{S_1} = \CP{^2}{\phi^2}$, если $n=2$), который зависит только от угловых переменных.

Таким образом, 
\[
  \Delta\E_n(x) = \begin{cases}
    \frac{(2-n)(1-n)}{(n-2)|S_1|}r^{-n} - \frac{(2-n)(n-1)}{(n-2)|S_1|}r^{-n} = 0,&n\ge 3;\\
    -\frac1{2\pi}\frac1{r^2} + \frac1{2\pi}\frac1{r^2} = 0,&n=2.
\end{cases}
\]
Отсюда что следует? От всей формулы остаётся два слагаемых. Это позволяет утверждать, что 
\[
  \big(\Delta\E_n(x),\phi(x)\big) = \yo\e0\bigg(\Gint{\dl(B_R\dd B_\e)}\E_n\CP{\phi}{\VE\nu}\,dS - 
    \Gint{\dl(B_R\dd B_\e)}\CP{\E_n}{\VE\nu}\phi\,dS\bigg).
\]
Несложно увидеть, что $\supp\phi\subset B_R$, то есть $\phi=0$ в окрестности $S_R$. А $\dl(B_R\dd B_\e) = S_R\cup S_\e$.
\begin{equation}\label{FROL4}
  \bigg|\Gint{\dl(B_R\dd B_\e)}\E_n\CP{\phi}{\VE\nu}\,dS \bigg|\le \Gint{S_\e}|\E_n|\left|\CP\phi{\VE\nu}\right|\,dS\le|\E_n|\big|_{S_\e}\sup\limits_{S_\e}|\nabla\phi||S_\e|,
\end{equation}
где $|S_\e|$ "--- $(n-1)$-мерный объём сферы $S_\e$.
\[
  |S_\e|\cdot |\E_n|\big|_{S_\e} = \begin{cases}
    \frac1{(n-2)|S_1|}\frac1{\e^{n-2}}\cdot |S_1|\e^{n-1},&n\ge3;\\
    \frac1{2\pi}|\ln\e|\e \cdot 2\pi \e,&n=2.
\end{cases}
\]
Поэтому $|S_\e|\cdot |\E_n|\big|_{S_\e}\to0$ при $\e\to+0$. И мы будем иметь
\[
  \bigg|\Gint{\dl(B_R\dd B_\e)}\E_n(x)\CP{\phi}{\VE\nu}(x)\,dS\bigg|\to0\pau \e\to0.
\]
На что надо обратить было внимание: $\sup\limits_{S_{\e}}|\nabla\phi|\le\|\phi\|_{C^1(\R^n)}<\infty$. Ну и всё, устремляем правую часть неравенства \eqref{FROL4} к нулю.

Посмотрим, как ведёт себя другой интеграл. Нам нужно сосчитать такую производную на сфере радиуса $\e$ (нормаль направлена внутрь сферы, чтобы быть внешней по отношению к области). Нормаль коллинеарная радиусу, но производная по радиусу имеет другой знак. (Обратим внимание, что $|S_1|=2\pi$ для $n=2$.)
\[
  \CP{\E_n}{\VE\nu}\bigg|_{S_\e} = - \CP{}r\begin{cases}
    \frac1{(n-2)|S_1|}\frac1{r^{n-2}}\bigg|_{r=\e},&n\ge3;\\
    \frac1{2\pi}\ln r\bigg|_{r = \e},&n=2
  \end{cases} = 
  \frac1{|S_1|\e^{n-1}}
\]
Там самым, аналогичное выражение для второго слагаемого
\[
  \Gint{\dl(B_R\dd B_\e)}\CP{\E_n}{\VE\nu}\phi\,dS = \Gint{S_\e}\CP{\E_n}{\VE\nu}\phi\,dS = \frac1{|S_\e|}\Gint{S_\e}\phi\,dS
\]
К чему стремится последнее? Несложно увидеть, что к $\phi(0)$, но как это строго доказать? В самом деле, напишем 
\[
  \frac1{|S_\e|}\Gint{S_\e}\phi\,dS - \phi(0) = \frac1{|S_\e|}\Gint{S_\e}\big(\phi(x)-\phi(0)\big)\,dS.
\]
Поэтому 
\[
  \bigg|\frac1{|S_\e|}\Gint{S_\e}\phi\,dx - \phi(0)\bigg|\le \frac1{|S_\e|}\Gint{S_\e}\big|\phi(x) - \phi(0)\big|\,dS\le\sup\limits_{x\in S_\e}\big|\phi(x) - \phi(0)\big|.
\]

Все наши умозаключения позволяют прийти вот к такому вот выводу.
\[
  \big(\Delta\E_n(x),\phi(x)\big) = \phi(0)
\]
или, иными словами, что и требовалось доказать.
\end{Proof}

\subsection{Фундаментальное решение оператора теплопроводности}
Громоздкие формулы запоминать не надо.

Есть некие методы получать фундаментальные решения. Методы достаточно громоздкие и требуется предварительная теория преобразований Фурье для обобщённой функции. Мы поэтому будем лишь смотреть на результаты.
\begin{The}
  Пусть
  \[
     \E(x,t) = \frac{\theta(t)}{(2a\sqrt{\pi t})^n}e^{-\frac{|x|^2}{4a^2t}},
  \]
  где
  \[
  \heviside[t]
  \]
Тогда
\[
  \E_t - a^2\Delta\E = \delta(x,t),
\]
то есть $\E(x,t)$ является фундаментальным решением оператора теплопроводности $\dl_t-a^2\Delta$.
\end{The}
Очень трудно угадать, что фундаментальное решение будет именно такое. Есть некая непростая процедура, которую кто-то когда-то проделал, например, лесница Ферма. Если её проделать для оператора теплопроводности, это займёт больше одного целого дня. Зная же результат, мы можем просто доказать.
\begin{Proof}
  Пусть $\phi\in\D(\R^{n+1})$ (есть ещё переменная $t$). Имеем
\[
   (\E_t - a^2\Delta\E,\phi) = (\E,-\phi_t-a^2\Delta\phi) = 
\]
У нас $\E$ локально суммируемая функция. Единственная проблема при $t=0$. Знаменатель стремится к нулю, а в числителе стоит экспонента, которая тоже стремится к нулю и причём быстрее. Отсюда следует вот что, что наше выражение есть ничто иное, как интеграл
\[
  = \Gint{\R^{n+1}} \E(-\phi_t-a^2\Delta\phi)\,dx\,dt =
    - \int\limits_0^\infty\Gint{\R^n}\frac{e^{-\frac{|x|^2}{4a^2t}}}{(2a\sqrt{\pi t})^n}(\phi_t-a^2\Delta\phi)\,dx\,dt=
\]
При перебрасывании производной по $t$ возникает проблема в нуле, там особенность. Поэтому будем переходить к пределу по $t$, а по $x$ будем брать интеграл по достаточно большому шару $B_R$.
\[
  = - \yo\e{+0}\int\limits_\e^\infty\Gint{B_R}\frac{e^{-\frac{|x|^2}{4a^2t}}}{(2a\sqrt{\pi t})^n}(\phi_t-a^2\Delta\phi)\,dx\,dt,
\]
где $\supp\phi\subset B_R\times\R$.
У нас тут два интеграла написано. Первый
\[
  \yo\e{+0}\int\limits_\e^\infty\Gint{B_R}\frac{e^{-\frac{|x|^2}{4a^2t}}}{(2a\sqrt{\pi t})^n}\phi_t\,dx\,dt = 
    \Gint{B_R}\int\limits_\e^\infty \frac{e^{-\frac{|x|^2}{4a^2t}}}{(2a\sqrt{\pi t})^n}\phi_t\,dt\,dx = 
  \Gint{B_R} \frac{e^{-\frac{|x|^2}{4a^2t}}}{(2a\sqrt{\pi t})^n}\phi\bigg|_{t=\e}^\infty\,dx - 
    \Gint{B_R}\int\limits_\e^\infty\CP{}t\left(\frac{e^{-\frac{|x|^2}{4a^2t}}}{(2a\sqrt{\pi t})^n}\right)\phi\,dt\,dx.
\]
В этом выражении меня пока интересует первое слагаемое. Сосчитаем его
\[
  \Gint{B_R}\frac{e^{-\frac{|x|^2}{4a^2t}}}{(2a\sqrt{\pi t})^n}\phi\bigg|_{t=\e}^\infty\,dx  = 
  -\Gint{B_R}\frac{e^{-\frac{|x|^2}{4a^2\e}}}{(2a\sqrt{\pi \e})^n}\phi(x,\e)\,dx
\]
Попробуем выяснить, к чему это стремится, когда $\e\to0$. Это мы сегодня уже не успеем.

Пок что для второго слагаемого напишем следующее выражение
\[
  \int\limits_\e^\infty\Gint{B_R}\frac{e^{-\frac{|x|^2}{4a^2t}}}{(2a\sqrt{\pi t})^n}\Delta\phi\,dx\,dt = 
\]
При каждом фиксированном $t\pau \phi=0$ в окрестности $B_R$, поэтому
\[
  = \int\limits_\e^\infty \Delta\left(\frac{e^{-\frac{|x|^2}{4a^2t}}}{(2a\sqrt{\pi t})^n}\phi\right)\phi\,dx\,dt.
\]
Таким образом,
\[
  \int\limits_\e^\infty \Gint{B_R}\frac{e^{-\frac{|x|^2}{4a^2t}}}{(2a\sqrt{\pi t})^n}(\phi_t + a^2\Delta\phi)\,dx\,dt = 
  - \Gint{B_R}\frac{e^{-\frac{|x|^2}{4a^2\e}}}{(2a\sqrt{\pi \e})^n}\phi(x,\e)\,dx + 
    \int\limits_0^\e\Gint{B_R}(\dl_t-a^2\Delta) \frac{e^{-\frac{|x|^2}{4a^2t}}}{(2a\sqrt{\pi t})^n}\phi\,dx\,dt.
\]
В следующий раз перейдём к пределу и всё покажем.
\end{Proof}
