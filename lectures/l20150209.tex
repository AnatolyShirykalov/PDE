\section{Задача Коши для уравнения теплопроводности}
Задача Коши для уравнения теплопроводности имеет вид
\begin{equation}\label{KouchiTaskTemp}
  \begin{cases}
     u_t = a^2\Delta u + f(x,t),& x\in\R^n,\ t>0,a=\const>0;\\
     u(x,0) = u_0(x).
  \end{cases}
\end{equation}
Мы будем понимать обобщённую задачу Коши, как в учебнике Владимирова, то есть решение из класса $\D'(\R^n)$.

Но всё началось с классической задачи.
\begin{Def}
  Обозначим за $\M$ множество измеримых функций $f$, таких, что во-первых, 
\[\forall\ T>0\pau f\in \L_{\infty}\big(\R^N\times[0,T]\big),\]
и при этом $f=0$ почти всюду на промежутке $\R^n\times(-\infty,0)$
\end{Def}
То есть множеством определения является $\R^{n+1}$, они существенно ограничены на любом ограниченном промежутке справа от нуля и эквивалентны нулевой функции на отрицательной полуоси по последнему аргументу.

Сколько нужно гладкостей для решения? Нужно дифферецировать по иксам, по $t$, и нужно выполнение равенства $u(x,0)=u_0(x)$. Вроде бы этого достаточно. Но тут есть такая деталь. В тридцатые годы академик Тихонов заметил, что если рассмотреть задачу $u_t = \Delta u$ и $u(x,0)=0$, то ноль очевидно решение. Но ведь есть и ещё решения нетривиальные. Физически эта ситуация абсурдна, ведь такая задача описывает нагревания бесконечно длинного стержня. Получается, что и без источника тепла стержень может нагреться. Тихонов обратил на это внимание. Как преодолеть эту ситуацию? Мы не будем ловить точность. Мы будем использовать класс $\M$, его достаточно. Ещё, мы потребуем немножно уже класс дифференцирования.
\begin{Def}
  Классическим решением задачи Коши \ref{KouchiTaskTemp} называется функция\footnote{По $t$ избыточное требование на дифференцируемость. Но мы будем пользоваться такой устоявшейся подстановкой. Здесь следует обратить внимание, что дифференцируемость требудется на открытом множестве.}
\[
  u\in M\cap C^2\big(\R^n\times(0,\infty)\big)\cap C\big(\R^n\times[0,\infty)\big),
\]
такая что выполнено\footnote{В классическом смысле дифференцирования, как вас учили на первом курсе.} \eqref{KouchiTaskTemp}.
\end{Def}

Мы хотим доказывать существование и единственность решения. Это удобнее делать для обобщённых решений. Мы сначала всё для них сделаем, а потом покажем, как это связано с классическими решениями.

Пусть в \eqref{KouchiTaskTemp} $u\in C^{2,1}\big(\R^n\times(0,\infty)\big)\cap
C\big(\R^n\times[0,\infty)\big)$ (два раза дифференцируема по $x$, один раз по $t$), $f\in C\big(\R^n\cap(0,\infty)\big)$, $u_0\in C(\R^n)$. Положим 
\[
  \Til u(x,t) = 
  \begin{cases}
    u(x,t),&x\in\R^n,\ t\ge0\\
    0&,x\in\R^n,t<0.
  \end{cases}
\]
И подставим $\Til u$ в уравнение, производя обобщённое дифференцирование. Для любого $\phi\in\D(\R^{n+1})$ имеем
\begin{multline*}
  (\Til u_t - a^2\Delta\Til u,\phi) = 
  (\Til u,-\phi_t - a^2\Delta\phi) =
  -\int\limits_{\R^{n+1}}\Til u\,\phi_t\,dx\,dt - 
  a^2\int\limits_{\R^{n+1}}\Til u\Delta\phi\,dx\,dt = \\
  = - \int\limits_{\R^n}\int\limits_0^\infty u\,\phi_t\,dt\,dx - 
    a^2\int\limits_0^\infty\int\limits_{\R^n}u\,\Delta\phi\,dx\,dt=
    \cmt{Формула у нас такая $\int\limits_a^bh'\,g\,dt = fg\big|_a^b - \int\limits_a^bf\,g'\,dt$}\\
    = \Gint{\R^n}\bigg(u(x,0)\,\phi(x,0) + \int\limits_0^\infty u_t\,\phi\,dt\bigg)\,dx -
      a^2\int\limits_0^\infty \Gint{\R^n}\Delta u\,\phi\,dx\,dt = 
      \Gint{\R^n}u_0(x)\,\phi(x,0)\,dx + 
      \int\limits_0^\infty\Gint{\R^n}\underbrace{(u_t - a^2\Delta u)}_{=f}\phi\,dx\,dt=\\
  = \Gint{\R^n}u_0(x)\,\phi(x,0)\,dx
    + \int\limits_0^\infty\Gint{\R^n} f(x,t)\,\phi(x,t)\,dx\,dt = \\
 \cmt{запишем это через прямое произведение обощённых функций $u_0(x),\delta(t)$.}\\
    =\big(u_0(x)\delta(t),\phi(x,t)\big) + 
    +\big(\Til f(x,t),\phi(x,t)\big)
\end{multline*}
где $ \Til f$ "--- локально суммируемая функция, определённая формулой
\[
  \Til f(x,t)=\begin{cases}
    f(x,t),&x\in\R^n,\ t\ge0;\\
    0,&x\in\R^n,\ t<0.
  \end{cases}
\]
Таким образом,
\begin{equation}\label{OBOB}
  \Til u_t - a^2\Delta \Til u = \Til f(x,t) + u_0(x)\,\delta(t).
\end{equation}
Вот, что получается из классического решения. Теперь всё готово для того, чтобы дать определение обобщённого решения.
\begin{Def}
  Обобщённым решением задачи Коши \eqref{KouchiTaskTemp} называется функция $\Til u\in\M$, удовлетворяющая соотношению \eqref{OBOB}.
При этом мы предполагаем, что $\Til f\in\M$ и $u_0\in\L_\infty(\R^n)$.
\end{Def}
Найти решение уравнения \eqref{OBOB} нам проще, чем решить задачу Коши. Там надо было искать определённый класс гладкости, смотреть, что выполнено начальное условие, ещё были другие требования. Здесь у нас только одно уравнение. Оказывается, что в таком классе решения, если существуют, то единственные. Спрашивается, при каких условиях? Вот при тех, которые мы написали
\begin{itemize}
\item $\Til u\in\M$;
\item $\Til f\in\M$;
\item $u_0\in\L_\infty(\R^n)$.
\end{itemize}

Как показать это? У нас метод один: взять фундаментальное решение, доказать, что существует свёртка. Сформулируем результат в виде теоремы.
\begin{The}
  Пусть 
\begin{itemize}
\item $\Til f\in\M$;
\item $u_0\in\L_\infty(\R^n)$.
\end{itemize}
Тогда обобщённая задача Коши имеет, причём единственное, решение $\Til u\in\M$. Более того, мы можем написать даже формулу, которая позволяет сосчитать обобщённое решение: формула Пуассона
\begin{equation}\label{OBOBTemp}
  \Til u(x,t) = \theta(t)\int\limits_0^t\Gint{\R^n}
    \frac{f(\xi,\tau)}{2\,a\sqrt{\pi(t-\tau)}}
    e^{-\frac{|x-\xi|^2}{4\,a^2(t-\tau)}}\,d\xi\,d\tau + 
    \frac{\theta(t)}{(2\,a\sqrt{\pi t})^n}
    \Gint{\R^n} u_0(\xi)\,e^{-\frac{|x-\xi|^2}{4\,a^2t}}\,d\xi.
\end{equation}
Обратите внимание, что все интегралы здесь корректно определены. В любой конечной полосе по $t$ функции $f$ ограничены, значит, первый интеграл сходится.
\end{The}
\begin{Proof}
  Покажем, что существует свёртка $\Til u$
\[
  \E(x,t)\star\big(\Til f(x,t)+u_0(x)\,\delta(t)\big),
\]
где
\[
  \E(x,t) = \frac{\theta(t)}{(2\,a\,\sqrt{\pi\,t})^n}e^{-\frac{|x|^2}{4\,a^2\,t}}
\]
есть фундаментальное решение уравнения (или, можно сказать, оператора) теплопроводности. 

В самом деле, пусть $\eta_k\in\D(\R^{2n+2})$, $k=1,2,\dots$ "--- компактное исчерпание единицы. Имеем $\forall\ \phi\in\D(\R^{n+1})$
\begin{multline*}
  \yo k\infty
   \Big(\E(x,t)\big(\Til f(\xi,\tau) +\Til u_0(\xi)\,\delta(\tau)\big),
        \phi(x+\xi,t+\tau)\,\eta_k(x,t,\xi,\tau)\Big) = \\
  = \yo k\infty
   \big(\E(x,t)\Til f(\xi,\tau),\phi(x+\xi,t+\tau)\eta_k(x,t,\xi,\tau)\big) +
  \yo k\infty
    \big(\E(x,t)\,\Til u_0(\xi)\,\delta(\tau),\phi(x+\xi,t+\tau)\eta_k(x,t,\xi,\tau)\big).
\end{multline*}
Заметим, что $\Til f$ локально суммируемая функция. Значит, следующую скобку можем заменить на интеграл.
\begin{multline*}
  \big(\E(x,t)\Til f(\xi\tau),\phi(x+\xi,t+\tau)\eta_k(x,t,\xi,\tau)\big) = \\ =
  \int\limits_0^\infty
  \int\limits_0^\infty
  \Gint{\R^n}
  \Gint{\R^n}
  \E(x,t)\Til f(\xi,\tau)\phi(\underbrace{x+\xi}_{y},\underbrace{\vphantom{\xi}t+\tau}_{s})\eta_k(x,t,\xi,\tau)\,dx\,\xi\,dt\,d\tau = \\
  \cmt{$x=y-\xi$, $t = s-\tau$ "--- простейшая линейная замена. Каждый модуль якобиана единица.}\\
  = 
  \Gint{\R^n}
  \Gint{\R^n}
  \int\limits_0^\infty
  \int\limits_0^\infty
  \E(y-\xi,s-\tau)\Til f(\xi,\tau)\phi(y,s)\eta_k(y-\xi,s-\tau,\xi,\tau)\,ds\,d\tau\,dy\,d\xi = \\
  = 
  \Gint{\R^n}
  \Gint{\R^n}
  \int\limits_0^\infty
  \int\limits_0^\infty
  \frac{\theta(s-\tau) 
  e^{-\frac{|y-\xi|^2}{4\,a^2(s-\tau)}}}
  {(2\,a\,\sqrt{\pi(s-\tau)})^n}
  \Til f(\xi,\tau)\phi(y,s)\eta_k(y-\xi,s-\tau,\xi,\tau)\,ds\,d\tau\,dy\,d\xi=\\
  \cmt{Может ли $s>\tau$? там мы интегрируем ноль, $\theta=0$ при отрицательных аргументах}\\
  = \int\limits_0^\infty \Gint{\R^n} \phi(y,s)\,dy\,ds\int\limits_0^\infty\Gint{\R^n}
  \frac{\theta(s-\tau) 
  e^{-\frac{|y-\xi|^2}{4\,a^2(s-\tau)}}}
  {(2\,a\,\sqrt{\pi(s-\tau)})^n}
  \Til f(\xi,\tau)\eta_k(y-\xi,s-\tau,\xi,\tau)\,d\tau\,d\xi=\\
\cmt{интегрирование по $\tau$ могу проводить не до $\infty$, а до $s$, но не совсем}
\end{multline*}
Перепишем интеграл отдельно
\begin{multline*}
  \int\limits_0^\infty \Gint{\R^n}
  \frac{\theta(s-\tau) 
  e^{-\frac{|y-\xi|^2}{4\,a^2(s-\tau)}}}
  {(2\,a\,\sqrt{\pi(s-\tau)})^n}
  \Til f(\xi,\tau)\eta_k(y-\xi,s-\tau,\xi,\tau)\,d\tau\,d\xi=\\
  \theta(s)\int\limits_0^s
  \frac{e^{-\frac{|y-\xi|^2}{4\,a^2(s-\tau)}}}
  {(2\,a\,\sqrt{\pi(s-\tau)})^n}
  f(\xi,\tau)\,\eta_k(y-\xi,s-\tau,\xi,\tau)
\,d\xi\,d\tau\to\\
  \to \theta(s)\int\limits_0^s
  \frac{e^{-\frac{|y-\xi|^2}{4\,a^2(s-\tau)}}}
  {(2\,a\,\sqrt{\pi(s-\tau)})^n}
  \Til f(\xi,\tau)\,d\xi\,d\tau
\end{multline*}
по теореме Лебега об ограниченной сходимости. ($\theta(s)$ появилась чисто формально, чтобы при $s<0$ точно получился ноль, ведь $\int\limits_0^s$ при отрицательных $s$ определён и не обязан быть нулём.)

Давайте покажем, что теорема об ограниченной сходимости здесь действительно применима, то есть укажем мажоранту.
\begin{multline*}
  \int\limits_0^s\Gint{\R^n}
  \frac{e^{-\frac{|y-\xi|^2}{4\,a^2(s-\tau)}}}
  {(2\,a\,\sqrt{\pi(s-\tau)})^n}
  \underbrace{\big|\Til f(\xi,\tau)\big|}_{\Til f\in\L_\infty(\R^n\times(0,s))}
  \big|\eta_k(y-\xi,s-\tau,\xi,\tau)\big|\,d\xi\,d\tau\le
  \const\int\limits_0^s\Gint{\R^n}
  \frac{e^{-\frac{|y-\xi|^2}{4\,a^2(s-\tau)}}}
  {(2\,a\,\sqrt{\pi(s-\tau)})^n}\,d\xi\,d\tau = \\
  \left\{
     \begin{matrix}
	\zeta = \frac{y-\xi}{2\,a\,\sqrt{s-\tau}}\\
	\xi = -2a\,\sqrt{s-\tau}\,\zeta + y\\
	d\zeta = 2\,a\big(2\,a\sqrt{s-\tau}\big)^n\,d\zeta
     \end{matrix}
  \right\}=
\cmt{это всё векторные выражения, якобиан считаем}\\
  =\frac1{\pi^{n/2}}\int\limits_0^s\underbrace{\Gint{\R^n}
    e^{-|\zeta|^2}\,d\zeta}_{<\infty}\,d\tau.
\end{multline*}
Теперь я могу написать то, чего и хотел
\begin{multline*}
  =
   \left(\theta(s)\int\limits_0^s\Gint{\R^n}
  \frac{e^{-\frac{|y-\xi|^2}{4\,a^2(s-\tau)}}}
  {(2\,a\,\sqrt{\pi(s-\tau)})^n}
  f(\xi,\tau)\,\eta_k(y-\xi,s-\tau,\xi,\tau)\,d\xi\,d\tau,\phi(y,s)
  \right)\\
\end{multline*}
Мы получили с точностью до обозначений первое слагаемое формулы Пуассона. Всё остальное будет точно так же.

Сейчас второе слагаемое получим. Со вторым слагаемым связано более озорное выражение, там есть некая функция, которая не является локально суммируемой.

Слово Россия было впервые написано греческим
\[
 \rho\w\sigma \iota\alpha\zeta
\]
А русские первые князья говорили на староскандинавском. И вот эти греческие договоры написаны на старостандинавском и на греческом. Происхождение слова «собака» до сих пор гадают.

\[
  \Big(\E(x,t)\star \big(u_0(x)\delta(t)\big),\phi(x,t)\Big) = 
  \yo k\infty\big(\E(x,t)\,u_0(\xi)\delta(\tau),\eta_k(x,t,\xi,\tau)\,
  \phi(x+\xi,t+\tau)\big).
\]
Надо доказать, что предел существует для любого компактного исчерпания единицы.  Нужно просто вспомнить определение прямого произведения. Нужно избавиться от $\delta$-функции. Умножения идут справа налево.
\begin{multline*}
  \Big(\E(x,t)\star \big(u_0(x)\delta(t)\big),\phi(x,t)\Big) = 
  \yo k\infty\big(\E(x,t)\,u_0(\xi)\delta(\tau),\eta_k(x,t,\xi,\tau)\,
  \phi(x+\xi,t+\tau)\big) = \\
  = \bigg(\E(x,t),
  \Big(u_0(\xi),\big(\delta(\tau),\eta_k(x,t,\xi,\tau)\phi(x+\xi,t+\tau)\big)\Big)\bigg)=\\
  = \Big(\E(x,t),\big( u_0(\xi),\eta_k(x,t,\xi,0)\phi(x+\xi,t)\big)\Big)=
  \bigg(\E(x,t),\Gint{\R^n} u_0(\xi)\,\eta_k(x,t,\xi,0)\phi(x+\xi,t)\,d\xi\bigg)=\\
\cmt{вспоминаем, что такое $\E(x,t)$. Ещё один интеграл будет, где вместо $\theta$ интегрируем от $0$ до $+\infty$}\\
  = \int\limits_0^\infty\Gint{\R^n}\frac{e^{-\frac{|x|^2}{4\,a^2\,t}}}{(2\,a\,\sqrt{\pi\,t})^n}\,dx\,dt
   \Gint{\R^n} u_0(\xi)\,\eta_k(x,t,\xi,0)\phi(x+\xi,t)\,d\xi=\\
\cmt{Моё искуство состоит только в том, как правильно расставить интегралы}\\
\left\{
  \begin{matrix}
    y = x+\xi\\
dx = dy\\
x = y-\xi
\end{matrix}
  \right\}=\\
=
\int\limits_0^\infty
\Gint{\R^n}\Gint{\R^n}
\frac{e^{-\frac{|y-\xi|^2}{4\,a^2\,t}}}{(2\,a\,\sqrt{\pi\,t})^n}
u_0(\xi)
\eta_k(y-\xi,t,\xi,0)
\phi(y,t)\,d\xi\,dy\,dt\to
  \int\limits_0^\infty\Gint{\R^n}\phi(y,t)\,dy\,dt\Gint{\R^n}
\frac{e^{-\frac{|y-\xi|^2}{4\,a^2\,t}}}{(2\,a\,\sqrt{\pi\,t})^n}
u_0(\xi)\,d\xi=
\end{multline*}
Я снова хочу применить (применил) теорему Лебега об ограниченной сходимости. В чём отличие от предыдущего случая? Один интеграл берётся по $t$, один по $y$, один по $\xi$. По $y$ и $t$ интегрируем по компактам-проекциям носителей на области изменения переменной интегрирования. Нужно при фиксированном $y$ и $t$ оценить интеграл по $\xi$.
Сделаем ещё один шаг в цепочки равенств, а потом я дам ещё один комментарий по поводу применения теоремы об ограниченной сходимости.
\[
  \bigg(\frac{\theta(t)}{(2\,a\,\sqrt{\pi\,t})^n}\Gint{\R^n}e^{-\frac{|y-\xi|^2}{4\,a^2\,t}}\,u_0(\xi)\,d\xi,\phi(y,t)\bigg)
\]
Это есть ни что иное, как второе слагаемое в формуле Пуассона.

А теперь покажем, как предельный переход осуществили, применяя теорему Лебега об ограниченной сходимости. Мы должны взять всё подынтегральное выражение и построить для него мажоранту. То, что вот это всё сходится почти всюду вот к тому, что у вас получается, когда $\eta_k$ выбросите, значит, нет никаких сомнений. Это просто следует из того, что $\eta_k$ "--- это компактное исчерпание единицы. Давайте построим мажоранту, то есть функцию, которая от $k$ не зависит.
\begin{multline*}
  \int\limits_0^\infty\Gint{\R^n}\Gint{\R^n}
  \frac{e^{-\frac{|y-\xi|^2}{4\,a^2\,t}}}{(2\,a\,\sqrt{\pi\,t})^n}\,
  \underbrace{\big|u_0(\xi)\big|}_{\in\L_\infty(\R^n)}\,
  \underbrace{\big|\phi(y,t)\big|}_{\in\D^{n+1}}\,
  \underbrace{\big|\eta_k(y-\xi,t,\xi,0)\big|}_{\text{комп.\,исч.\,ед.}}
  \,d\xi\,dy\,dt\le\\
\le \const\Gint{\supp\phi}dy\,dt\Gint{\R^n}
\frac{e^{-\frac{|y-\xi|^2}{4\,a^2\,t}}}{(2\,a\,\sqrt{\pi\,t})^n}\,d\xi=\\
\left\{
\begin{matrix}
  \zeta = \frac{y - \xi}{2\,a\,\sqrt t}\\
  \xi = y - 2\,a\,\sqrt t\,\zeta\\
  d\xi = (2\,a\,\sqrt t)^n\,d\zeta
\end{matrix}
\right\}\le\\
\const\Gint{\supp \phi}\,dy\,dt\Gint{\R^n}e^{-|\zeta|^2}\,\zeta<\infty.
\end{multline*}
Таким образом, в качестве мажоранты в теореме Лебега об ограниченной сходимости возьмём
\[
  F(y,\xi,t) = \const\chi_{\supp\phi}(y,t)\frac{e^{-\frac{|y-\xi|^2}{4\,a^2\,t}}}{(2\,a\,\sqrt{\pi\,t})^n}.
\]
Тем самым получили мы формулу Пуассона.


Надо показать, что функция, определяемая формулой Пуассона, лежит в классе $\M$. Очевидно
\[
  u(x,t) = \theta(t)\int\limits_0^t\Gint{\R^n}\frac{f(\xi,\tau)}{(2\,a\,\sqrt{\pi(t-\tau)})^n} e^{-\frac{|x-\xi|^2}{4\,a^2(t-\tau)}}\,d\xi\,d\tau + 
  \frac{\theta(t)}{(2\,a\,\sqrt{\pi\,t})^n}\Gint{\R^n} u_0(\xi)
  e^{-\frac{|x-\xi|}{4\,a^2\,t}}\,d\xi\in\M.
\]

Это останется в качестве упражнения. Делается с помощью той же заменой, что и оценки для теореме об ограниченной сходимости, но здесь нужно доказть, что $u\in\L_\infty\big(\R^n\times(0,T)\big)$ для любого $0<T<\infty$. Воспользуйтесь заменой
\[
  \zeta = \frac{x-\xi}{2\,a\,\sqrt{t-\tau}};\qquad \zeta = \frac{x-\xi}{4\,a\,\sqrt t}.
\]
\end{Proof}
