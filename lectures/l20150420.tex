\section{Гармонические функции}

Строгое определение такое
\begin{Def}
  Пусть $\Omega$ "--- открытое непустое множество в $\R^n$, $n\ge 1$. Функция $u\in \D'(\Omega)$ называется гармонической в $\Omega$, если $\Delta u=0$ в $\Omega$.

Здесь $\Delta = \RY i1n\CP{^2}{x_i^2}$.
\end{Def}

Обычно дают определение $u\in C^2$ на гладкость. Но если требовать так, то она будет и бесконечно гладкой.

\begin{The}
  Пусть $u\in \D'(\Omega)$, причём $\Delta u = 0$ в $\Omega$. Тогда $u\in C^\infty(\Omega)$.
\end{The}
\begin{Proof}
Доказательство опирается на промежуточную лемму
\begin{Lem}
  Пусть $f\in \D'(\Omega)$, $k\in D(\Omega\times \Omega)$. Тогда
\[
  \Gint\Omega \big(f(x),k(x,y)\big)\,dy = 
  \bigg(f(x),\Gint{\Omega} k(x,y)\,dy\bigg)
\]
То есть утверждение состоит в том, что мы можем вносить интеграл внутрь.
\end{Lem}

\begin{proof}
  Доказательство будет основано на линейности обобщённой функции. Что такое интеграл? Берём область, разбиваем на конечное число маленьких областей, берём в каждой по точке и суммируем произведения значений функции на площадь маленькой области. По линейности функционала эту сумму мы можем задащить внутрь действия обобщённой функции. А предельный переход обосновывается непрерывностью функционала.

Как строго поступить? Мы знаем, что выражение
\[
  \big(f(x),k(x,\cdot)\big)\in \D(\Omega).
\]
Почему гладкость, ясно. А почему носитель компактный? Возьмём спроецируем компакт на один экземпляр $\Omega$. Проекция компакта компакт.
% картинка
% оси Oxy
% в первой четверти облачко (выпуклое для простоты)
% горизонтальные и вертикальные касательные к облачку от облачка до осей.

Раз выражение из $\D(\Omega)$, то интеграл в левой части рассматриваемого выражения существует.

Продолжим $k$ нулём на такое множество $\R^{2n}\dd\supp k$. Получим функцию $k\in \D(\R^{2n})$. Зачем? Для того, чтобы потом взять куб $Q_a$ с центром в нуле и ребром $a>0$ таким, что $\supp k\subset Q_a$. Ну и прекрасно. Дальше будем разбивать этот куб на маленькие кубики.
% рисунок
% квадрат с равномерной сеткой внутри и тем же облачком

Разобъём $Q_a$ на $N^m$ непересекающихся кубиков с ребром $\frac aN$. Обозначим эти кубики $\kubic _i$, $i=1,\dots,N^m$. Пусть $y_i\in \kubic_i$ "--- некоторые точки, $|\kubic_i|$ "--- объём $i$-го кубика. Тогда
\begin{multline*}
  \Gint\Omega\big(f(x),k(x,y)\big)\,dy = 
  \RY i1{N^m} \Gint{\kubic_i}\big(f(x),k(x,y)\big)\,dy = \\
  = \RY i1{N^m}\big(f(x),k(x,y)\big)|\kubic_i| + \RY i1N \Gint{\kubic_i}
    \big(f(x),k(x,y) - k(x,y_i)\big)\,dy=\\
\cmt{Нужно показать, что второе слагаемое стремится к нулю.}\\
= \bigg(f(x),\RY i1{N^m} k(x,y_i)|\kubic_i|\bigg) + 
	\RY i1{N^m}\Gint{\kubic_i} \big( f(x),k(x,y) - k(x,y_i)\big)\,dy = \\
  = \bigg(f(x),\RY i1{N^m} \Gint{\kubic_i}k(x,y)\,dy 
	+ \RY i1{N^m}\Gint{\kubic_i} \big(k(x,y_i) - k(x,y)\big)\,dy\bigg)
	+
	\RY i1{N^m}\Gint{\kubic_i} \big( f(x),k(x,y) - k(x,y_i)\big)\,dy=\\
 = \bigg( f(x),\Gint\Omega k(x,y)\,dy\bigg) + 
	\bigg(f(x),
		\RY i1{N^m}\Gint{\kubic_i}\big(k(x,y_i)-k(x,y)\big)\,dy
	\bigg) + 
	\RY i1{N^m}\Gint{\kubic_i} \big(f(x),k(x,y) - k(x,y_i)\big)\,dy.
\end{multline*}
Последнее слагаемое оцениваится по гладкости $k$. А для оценки второго нужно показать, что 
\[
	\phi_N(x)=	\RY i1{N^m}\Gint{\kubic_i}\big(k(x,y_i)-k(x,y)\big)\,dy \xrightarrow
[N\to\infty]
{\D(\Omega)}0
\]
Нужно вспомнить определение этой сходимости, не более того. Имеем
\begin{roItems}
\item $\supp\phi_N\subset$ проекции носителя $\supp k$ на область изменения переменной $x$, что от $N$ не зависит.
\item $\|\phi_N\|_{C^m(\Omega)} \to 0$ при $N\to \infty$ для всех $m$. В самом деле,
\[
  \dl^\alpha \phi_N(x) = \RY i1N \Gint{\kubic_i} \big(\dl_x^\alpha k(x,y_i) - \dl^\alpha_x(x,y)\big)\,dy.
\]

Всё свелось к дифференцированию интеграла по параметру. Возьмём вот такой модуль
\[
  \big|\dl^\alpha\phi_N(x)\big|\le
  \RY i1{N^m} \Gint{\kubic_i} \big|\dl_x^\alpha k(x,y_i) - \dl^\alpha_x k(x,y)\big|\,dy.
\]
У нас функция $k$ непрерывна со своими производными на своём компактном носителе, значит, она на нём равномерно непрерывна со всеми своими производными. Значит, 
\[
\forall\ \e>0\pau\exists\ \Til N\colon \forall\ N>\Til N\pau \big|k(x,y) - k(x,y_i)\big|<\e\]
 для $y,y_i\in \kubic_i$.
Таким образом, $\forall\ \e>0\pau \exists\ N\colon$ 
\[
  \big|\dl^\alpha\phi_N(x)\big|\le
  \e \RY i1{N^m} \Gint{\kubic_i}\,dy = \e|Q_a|.
\]
\end{roItems}
Таким образом, $\big(f(x),\phi_N(x)\big)\to0 $ при $N\to\infty$ в силу непрерывности $f(x)$ на пространстве $\D(\Omega)$. В то же время,
\begin{multline*}
  \RY i1{N^m}\Gint{\kubic_i} \big(f(x),k(x,y) - k(x,y_i)\big)\,dy = 
  \RY i1{N^m}\Gint{\kubic_i} \big(f(x),k(x,y)\big)\,dy -
  \RY i1{N^m}\Gint{\kubic_i} \big(f(x),k(x,y)\big)\,dy = \\
  =
  \Gint\Omega\big(f(x),k(x,y)\big)\,dy - 
    \RY i1{N^m} \big(f(x),k(x,y_i)\big)|\kubic_i| \to 0,
\end{multline*}
поскольку $\big(f(x),k(x,\cdot)\big)\in\D(\Omega)$ и поэтому интегрируема по $\Omega$ (что означает, что частичные суммы сходятся к интегралу.

Ну и всё, устремляя $N$ к бесконечности, мы получаем требуемое равенство.
\end{proof}

Теперь докажем теорему. Нужно доказать гладкость в области, то есть показать гладкость в окрестности каждой точки. Поэтому давайте так напишем.

Пусть $x\in\Omega$. Покажем, что $\exists\ \e>0\colon u\in C^{\infty}(B^x_{\e/2})$, Где $B_\e^x$ "--- шар радиуса $\e$ с центром в точке $x$.

Что мы будем делать? Воьмём срезку. Возьмём $\e$ настолько маленькое, что $B_{2\e}^x\in\Omega$. Пусть $\eta\in C^\infty_0(B_{2\e}^x)$, причём $\eta|_{B_\e^x} = 1$.
Имеем 
\[
  \Delta(\eta u) = \eta\Delta u + 2\nabla\eta\nabla u + \Delta\eta u = 2\nabla\eta\nabla u + \Delta\eta u = :f(x).
\]
Это как бином Ньютона. Но берётся прямо из определения оператора Лапласа.

Заметим, что $\supp( 2\nabla\eta\nabla u + \Delta\eta u)\subset \supp\eta$ и $\supp\eta u\subset\supp\eta$. Тогда я могу смело писать такое утверждение.

Дальше я могу поспользоваться теоремой единственности. Согласно это теореме получим
\begin{equation}\label{kauchuk}
  \eta u = \E_n\star (2\nabla\eta\nabla u + \Delta\eta u),
\end{equation}
где 
\[
  \E_n(x) = 
\begin{cases}
  \frac1{|S_1|(n-2)} |x|^{2-n},&n\ge3\\
  \frac1{2\pi}\ln|x|,&n=1.
\end{cases}
\]
Найдём свётку в правой части \eqref{kauchuk}. По определению свёртки имеем (одна из двух сворачиваемых обобщённых функций имеет компактный носитель) $\forall\ \phi\in\D(\R^n)$
\begin{multline*}
  \big(\E_n(x)\star f(x),\phi(x)\big) = \yo k\infty \big(\E_n(x)f(y),\lambda_k(x,y)\phi(x+y)\big)=\\
\cmt{Здесь $\lambda_k$ "--- компактное исчерпание единицы. У $f(y)$ компакнтый носитель.}\\
\cmt{Стабилизируется наша последовательность}\\
 = \yo k\infty\Big(\E_n(x),\big(f(y),\lambda_k(x,y)\phi(x+y)\big)\Big) 
 = \Big(\E_n(x),\big(f(y),\phi(x+y)\big)\Big) 
\end{multline*}
Если бы не точка $0$, то могли бы уже закончить доказательство. $\big(f(y),\phi(x+y)\big)\in \D(\R^n)$ по переменной $x$ для каждого $y$. Поэтому
\[
  \big(\E_n(x)\star f(x),\phi(x)\big) = \Gint{\R^n}\E_n(x) \big(f(y),\phi(x+y)\big)\,dy.
\]
В нуле у нас неприятная особенность у $\E_n(x)$. Как от неё избавиться?
\end{Proof}<++>
