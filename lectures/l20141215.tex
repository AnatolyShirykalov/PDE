\section{15 декабря}
Есть такое свойство у обобщённой функции, определённой на $\D(\R^{n+1})$, которая на самом деле зависит только от первых $n$ переменных, то есть зависимость от $x_{n+1}$ чисто формальная.

\begin{Def}
	Говорим, что $f(x)\in \D'(\R^{n+1})$ допускает продолжение $f_0(x')\in\D'(\R^n)$ на функции $\phi(x')\in \D(\R^n)$, $x = (x',x_{n+1})$, если для любого компактного исчерпания единицы $\eta_k\in\D(\R)$, $k=1,2,\dots$ имеет место соотношение
	\begin{equation}
	  \big(f_0(x'),\phi(x')\big) = \yo k\infty\big(f(x),\phi(x')\eta_k(x_{n+1})\big).
		\label{continuation}
	\end{equation}<++>
\end{Def}
Сразу возникает вопрос о корректности определения. Вот возьмём мы другое компактное исчерпание единицы, вдруг получится другой предел. Требуется, понять, почему предел не зависит от компактного исчерпания единицы. Это будет в качестве упражнения.
\begin{Task}
	Пусть для любого компактного исчерпания едницы существует предел в правой части \eqref{continuation}. Докажите, что этот предел не зависит от выбора $\eta_k$, $k=1,2,\dots$
\end{Task}

Приведу готовое решение в силу важности упражнения. Пусть $\eta_l\in\D'(\R)$ и $\lambda_k\in\D'(\R)$, $k=1,2,\dots$ "--- два компактных исчерпания единицы. Тогда $\eta_1,\lambda_1,\eta_2,\lambda_2,\eta_3,\lambda_3,\dots$ "--- тоже компактное исчерпания единицы.

\begin{The}
	Пусть $\L = \RY q1N\CP{ }{x_{n+1}^q}\L_q + \L_0$, где дифференциальный оператор
	\[
		\L_0 = \sum\limits_{|\a'|\le m_0}a_{\a'}\dl^{\a'},\quad \a' = (\a_1,\dots,\a_n),\pau \dl^{\a'} = \frac{\dl^{|\a'|}}{\dl x_1^{\a_1}\dots\dl x_n^{|\a_n|}},\pau |\a'| = \a_1+\dots+\a_n,
	\]
	и оператор
	\[
		\L_q = \sum\limits_{|\a|\le m_q}a_q \dl^\a,\quad \a = (\a_1,\dots,\a_n,\a_{n+1}),\pau \dl^{\a} = \frac{\dl^{|\a|}}{\dl x_1^{\a_1}\dots\dl x_{n+1}^{|\a_{n+1}|}},\pau |\a| = \a_1+\dots+\a_{n+1},
	\]
	все числа $a_{\a'},a_\a\in\C$.

	Предположим, что $\L u = f(x')\d(x_{n+1})$ и при этом $u(x)$ допускает продолжение $u_0(x')$ на пространстве основных функций $\D(\R^n)$, то есть 
	\[
		\big(u_0(x'),\phi(x')\big)= \yo k\infty\big(u(x),\phi(x')\eta_k(x_{n+1})\big)
	\]
	для любого компактного исчерпания единицы $\eta_k\in \D(\R)$, $k=1,\dots,n$.

	Тогда $\L_0 u_o(x') =f(x')$.
\end{The}
\begin{Proof}
	Пусть $\phi(x')\in\D(\R^n)$ и $\eta_k(x_{n+1})\in\D(\R)$, $k=1,2,\dots$ "--- компактное исчерпание единицы. Имеем
	\[
		\big(\L u,\phi(x')\eta_k(x_{n+1})\big) = \big(f(x')\d(x_{n+1}),\phi(x')\eta_k(x_{n+1})\big). 
	\]
	Несложно заметить, что такой вот предел
	\begin{multline*}
		\yo k\infty\big(f(x')\d(x_{n+1}),\phi(x')\eta_k(x_{n+1})\big) =
		\yo k\infty\big(f(x'),\phi(x')\big)\big(\d(x_{n+1}),\eta_k(x_{n+1})\big)=\\
		=\big(f(x'),\phi(x')\big)\yo k\infty\eta_k(0)=\big(f(x'),\phi(x')\big).
	\end{multline*}
	
	С другой стороны посчитаем предел
	\[
		 \big(\L u,\phi(x')\eta_k(x_{n+1})\big) = \Big(u(x),\L^*\big(\phi(x')\eta_k(x_{n+1})\big)\Big), 
	\]
	где $	 \L^* = \RY q1N(-1)^q\CP{ }{x_n^q}\L_q^* + \L_0^*$ "--- формально сопряжённый оператор к $\L$, где
	\[
		\L_q^* = \sum\limits_{|\a|\le m_q}(-1)^{|\a|}a_\a\dl^\a,\ q = 1,\dots, N,\quad
		\L_0^* = \sum\limits_{|\a'|\le m_q}(-1)^{|\a'|}a_{\a'}\dl^{\a'}.
	\]
	Теперь мы можем продолжить формулу (что там за $l_q$ и $M$ нам совершенно не важно)
	\[
		 \big(\L u,\phi(x')\eta_k(x_{n+1})\big) = \Big(u(x),\L^*\big(\phi(x')\eta_k(x_{n+1})\big)\Big) =
		 \eta_k(x_{n+1})\L^*_o\phi(x') + \RY q1M\DP{^q\eta_k(x_{n+1})}{x_n^q} \sum\limits_{|\a'|\le l_q} b_{\a'}\dl^{\a'}\phi(x')
	 \]
	 Здесь $b_{\a'}\in\C$, какие "--- совершенно не важно. Для удобства обозначим $\phi_q(x') = \sum\limits_{|\a'|\le l_q} b_{\a'}\dl^{\a'}\phi(x')$. Таким образом, получим
	 \[
		 \big(\L u(x),\phi(x')\eta_k(x_{n+1})\big) = \big( u(x),\eta_k(x_{n+1})\L_0^*\phi(x')\big) +
		 \RY q1M\bigg( u(x),\DP{^q\eta_k(x_{n+1})}{x_n^q}\phi_q(x')\bigg).
	\]
	При этом, очевидно, что 
	\[
		\yo k\infty \big(u(x),\eta_k(x_{n+1})\L_0^*(x')\big) = \big(u_0(x'),\L_0^*\phi(x')\big) = \big(\L_0 u(x'),\phi(x')\big),
	\]
	равняется тому, чему нужно для нашего уравнения. В то же время для любого $q = 1,2,\dots,M$ имеем
	\[
		\yo k\infty\bigg( u(x),\DP{^q\eta_k(x_{n+1})}{x_n^q}\phi_q(x')\bigg) = 0.
	\]
	Почему? Это в каком-то смысле у вас рассуждение уже встречалось, когда выписывали правило дифференцирования свёртки. Там для похожего предела надо было доказать, что предел равен нулю. В самом деле, обозначим 
	\[
		\lambda_k(x_{n+1}) = \eta_k(x_{n+1}) + \DP{^q\eta_k(x_{n+1})}{x_n^q}, \quad k=1,2,\dots
	\]
	Эти $\lambda_k$ дадут компактное исчерпание единицы. Тем самым пределы от выражения на $\eta_k$ и выражения на $\lambda_k$ должны равняться одному и тому же.
	\[
		\big(u_o(x'),\phi(x')\big) = \yo k\infty\big(u(x),\eta_k(x_{n+1})\phi_q(x')\big) = \yo k\infty\big(u(x),\lambda_k(x_{n+1})\phi_q(x')\big).
	\]
	Из этого равенства следует, что
	\[
		\cancel{\yo k\infty\big(u(x),\eta_k(x_{n+1})\phi_q(x')\big)} = \cancel{\yo k\infty\big(u(x),\eta_k(x_{n+1})\phi(x')} + \yo k\infty\bigg(u(x),\DP{^q\eta_k(x_{n+1})}{x_{n+1}^q}\big(x')\bigg).
	\]
	Откуда следует, что
	\[
		\yo k\infty\bigg( u(x),\DP{^q\eta_k(x_{n+1})}{x_n^q}\phi_q(x')\bigg) = 0.
	\]
	Последнее влечёт за собой
	\[
		\yo k\infty \RY q1M\bigg(u(x),\DP{^q\eta_k(x_{n+1})}{x^q_{n+1}}\phi_q(x')\bigg) = 0.
	\]
	Поэтому будем иметь
	\[
		\yo k\infty\big(\L u,\phi(x')\eta_k(x_{n+1})\big) = \big(\L_0 u_0(x'),\phi(x')\big).
	\]
	Таким образом, получим
	\[
		\big(\L_0 u_0(x'),\phi(x')\big) = \big(f(x'),\phi(x')\big)
	\]
	или, другими словами, $\L u_0(x') = f(x')$, что и требовалось доказать.
\end{Proof}

Попробуем в качестве следствия получить фундаментельное решение оператора для $n=2$.
\subsection{Фундаментальное решение двумерного волнового оператора}
\begin{The}
	Пусть $\E_2(x,t) = \frac{\q(at - |x|)}{2\pi a\sqrt{a^2 t^2 - |x|^2}}$, где $x = (x_1,x_2)\in\R^2$. Тогда
	\[
		\wave_a\E_2(x,t) = \d(x,t),\quad \wave_a = \CP{^2}{t^2} - a^2\Delta,\pau \Delta = \CP{^2}{x_1^2}+\CP{^2}{x_2^2}.
	\]
	Другими словами, $\E_2(x,t)$ является фундаментальным решением двумерного волнового оператора.
\end{The}
Мы с вами знаем фундаментальное решение трёхмерного.
\begin{Proof}
	Мы знаем, что обобщённая функция
	\[
		\E_3(x,t) = \frac{\q(t)}{4\pi a^2 t}\d_{S_{at}}(x),\quad x = (x_1,x_2,x_3),
	\]
	является фундаментальным решением трёхмерного волнового оператора, то есть
	\[
		\CP{^2\E_3}{t^2} + a^2\bigg(\CP{^2\E_3}{x_1^2} + \CP{^2\E_3}{x_2^2} + \CP{^2\E_3}{x_3^2}\bigg) = \underbrace{\delta(t)\delta(x')\delta(x_3)}_{\d(x,t)},\quad x' = (x_1,x_2).
	\]
	Покажем, что $\E_3(x,t)$ допускает продолжение на пространство $\D(\R^3)$, причём это продолжение как раз есть функция $\E_2(x',t)$, то есть вот такое соотношение нужно доказать
	\[
		\yo k\infty\big(\E_3(x,t),\eta_k(x_3)\phi(x',t)\big) = \big(\E_2(x',t),\phi(x',t)\big).
	\]
	Действительно
	\[
		\big(\E_3(x,t),\eta_k(x_3)\phi(x',t)\big) = \bigg(\frac{\q(t)}{4\pi a^2 t}\d_{S_{at}}(x),\eta_k(x_3)\phi(x')\bigg) =
		\frac1{4\pi a^2}\int\limits_0^\infty\frac{dt}t\int\limits_{|x|=at}\eta_k(x_3)\phi(x',t)  \,dS.
	\]
	У нас $t$ фиксировано, значит, $\phi$ зависит только от $x'$. Поэтому интеграл можно так устроить. Есть элемент поверхности сферы. Его ожно спроецировать на сечение. Как будут связаны площади $dS$ и $dS'$ на $x$ и $x'$ соответственно? $dx'=dS' = dS\cos(\nu,x_3)$, где $\nu$ "--- внешняя нормаль. Теперь надо сосчитать этот самый $\cos(\nu,x_3)$. Нужно взять радиус $at$ и $\cos(\nu,x_3) = \frac{\sqrt{a^2 t^2 - |x'|^2}}{at}$. И теперь я могу написать, чему равно $dS$
	\[
		dS = \frac{at\, dS'}{\sqrt{a^2t^2-|x'|^2}}.
	\]
	Интегралы по верхней половинке сферы и по нижней одинаковы. Сосчитаем один и удвоим.
	\begin{multline*}
		\frac1{4\pi a^2}\int\limits_0^\infty\frac{dt}t\int\limits_{|x|=at}\eta_k(x_3)\phi(x',t)  \,dS = 
		\frac1{2\pi a}\int\limits_0^\infty dt\int\limits_{|x'|\le at}\frac{\eta_k(x_3)  \phi(x',t)}{\sqrt{a^2t^2 - |x'|^2}}\,dx'\to\\
		\to\frac1{2\pi a}\int\limits_0^\infty dt\int\limits_{|x'|\le at}\frac{\phi(x',t)}{\sqrt{a^2t^2 - |x'|^2}}\,dx' = 
		\big(\E_2(x',t),\phi(x',t)\big).
	\end{multline*}
	по теореме Лебега об ограниченной сходимости. Доказательство завершается применением предыдущей теоремы.
\end{Proof}
