\section{Задача Коши для обыкновенного дифференциального уравнения с постоянными коэффициентами}
В качестве $\L$ возьмём такой дифференциальный оператор
\[
 \L = \frac{d^m}{dx^m} + a_{m-1}\frac{d^{m-1}}{dx^{m-1}}+\dots+a_1\frac d{dx}+a_0.\quad a_s\in\C,\ s=0,1,\dots,m-1.
\]
А задача Коши формулируется так:
\[
  \L w = f(x),\pau w(0)=w_0,\dots, w^{(n-1)}(0) = w_{n-1},\pau w_s\in \C,\ s=0,1\dots,m-1.
\]
Для простоты будем считать, что $f(x)\in C^\infty(\R)$. Вы знаете из курса ОДУ, что у этой задачи имеется единственное решение. Коэффициенты не обязательно для этого должны быть постоянными, но для уравнения с постоянными коэффициентами существует единственное глобальное решение, определённое по всём $\R$.

Обозначим 
\[\Til w(x) = \begin{cases}
  w(x),&x>0;\\ 0,&x<0.
\end{cases};\qquad\Til f(x) = \begin{cases}
  f(x),&x>0;\\
  0,&x<0.
\end{cases}
\]

И применим к $\Til w$ оператор $\L$. Если будете считать классическую производную, то поймёте, что в нуле это сделать нельзя. Но вы можете интерпретировать $\Til w(x)$ как обобщённую функцию, и, значит, её можно дифференцировать.
\[
  (\L\Til w,\phi) = (w,\L^*\phi),
\]
где $\L^*$ "--- такой вспомогательный оператор. Как он устроен? Если вы берёте одну производную, минус вылезает:
\[
  \left(\frac{du}{dx},\phi\right) = -\left(u,\frac{d\phi}{dx}\right)
\]

Значит, $\L^* = (-1)^m\DP{^m}{x^m}+ (-1)^{m-1}d_{m-1}\DP{^{m-1}}{x^{m-1}}+\dots+a_1\DP{}x+a_0$. Он называется формально сопряжённым оператором. Слово «формально» появляется, так как мы не задумываемся об области определения оператора, интересуемся лишь символической записью.

Так как $\Til w\in L_{1,loc}(\R)$, то 
\begin{equation}\label{vsesuda1}
  (\Til w,\L^*\phi) = \int\limits_{-\infty}^\infty\Til w\L^*\phi\,dx = 
  \int\limits_0^\infty w\L^*\phi\,dx = 
  \RY s1m (-1)^s a_s\int\limits_0^\infty w\phi^{(s)}\,dx.
\end{equation}
Здесь одна тонкость, $s$ меняется от $0$ до $m$, а старший коэффициент единица. Значит, положим $a_m=1$.

Дальше я проделаю процедуру, называемую интегрированием по частям (учту ещё, что $\phi$ имеет компактный носитель, то есть на $\infty$ $\phi=0$)
\begin{multline*}
  \int\limits_0^\infty w\phi^{(s)}\,dx = w \phi^{s-1}\bigg|_0^\infty - \int\limits_0^\infty w'\phi^{(s-1)}\,dx = 
  = - w(0)\phi^{(s-1)}(0)-w'\phi^{s-2}\bigg|_0^\infty + \int\limits_0^\infty w''\phi^{s-2}\,dx = \\
  = -w(0) \phi^{(s-1)}(0)+w'(0) \phi^{(s-2)}(0)+\dots + (-1)^s w^{(s-1)}(0)\phi(0) + (-1)^s\int\limits_0^\infty s^{(s)}\phi\,dx.
\end{multline*}

Теперь мы подставим всё в сумму \eqref{vsesuda1} (учтём начальные условия $w^{(p)}(0) = w_p\in\C$)
\begin{multline*}
   (\Til w,\L^*\phi)  = \RY s1m(-1)^sa_s\RY p0{s-1}(-1)^{p+1} w^{(p)}(0)\phi^{(s-p-1)}(0) + \int\limits_0^\infty \underbrace{\RY s0m(-1)^sa_sw^{(s)}}_{\L w = f}\phi\,dx = \\
  = \RY s1m\RY p0{s-1}(-1)^{s+p+1}a_s w_p\underbrace{\phi^{(s-p-1)}(0)}_{(-1)^{s-p-1}\left(\delta^{(s-p-1)}(x),\phi(x)\right)}+\underbrace{\int\limits_0^{\infty} f\phi\,dx}_{(\Til f,\phi)} =\\
  =\bigg(\RY s1m \RY p0{s-1} a_s w_p \delta^{(s-p-1)}(x)+\Til f(x),\phi(x)\bigg),
\end{multline*}
откуда следует, что
\begin{equation}\label{vsesuda2}
  \L\Til w = \RY s1m\RY p0{s-1} a_sw_p\delta^{(s-p-1)}(x) + \Til f(x).
\end{equation}

В качестве примера рассмотрим вот такое уравнение $y'' + \w^2 y = f(x)$ с начальными условиями $y(0)=y_0$, $y'(0)= y_1$ ($\w>0$). Такое уравнение колебаний с правой частью. Положим
\[
  \Til y(x) = \begin{cases}y(x),&x\ge0,\\ 0,&x<0.\end{cases}
\]
Тогда, используя \eqref{vsesuda2}, получим $\Til y''+ \w^y=\delta(x) y_1 +\delta'(x)y_0+\Til f(x)$.

У нас была задача Коши. А мы получили одно уравнение, в которое входит всё.

\subsection{Фундаментальные решения обыкновенного оператора $\L$ с постоянными коэффициентами}
\begin{The}
  Пусть $W$ "--- решение задачи Коши
  \[
  \L w = 0,\pau w(0)=0,\dots, w^{(m-2)}(0)=0,\  w^{(m-1)}(0)=1.
\]
Тогда $\E(x) = \theta(x)W(x)$ "--- фундаментальное решение оператора $\L$, то есть $\L\E(x)=\delta(x)$.
\end{The}
\begin{Proof}
  Непосредственно следует из формулы \eqref{vsesuda2}.
\end{Proof}

В качестве примера возьмём оператор уже рассмотренного сегодня уравнения $\L = \DP{^2}{x^2} + \w^2$, $\w>0$. Рассмотрим задачу
\[
  w''+\w^2w = 0,\pau w(0)=0,\ w'(0)=1.
\]
Решение, как мы давно знаем, $W(x) = \frac1\w\sin\w x$. Фундаментальное решение
\[
 \E(x) = \frac{\theta(x)}\w\sin\w x.
\]

\subsection{Свёрточная алгебра}
Обозначим $\A = = \big\{f\in \D'(\R)\colon \supp f\subset [0,\infty)\big\}$.
\begin{Lem}
  Множество $\A$ образует коммутативную алгебру с единицей относительно свёртки и операций сложения и умножения на скаляр.
\end{Lem}
\begin{Proof}
Достаточно доказать, что $\forall\ f,g\in\A\pau \exists\ f\star g\in\A$.
Пусть $\eta_k$ "--- компактное исчерпание единицы. Возьмём такое $\tau\in C^{\infty}(\R)$, что $\tau \big|_{(-\infty,-1]}=0$, $\tau\big|_{[-1/2,\infty)}=1$. Положим $\tau_\e(x) = \tau\left(\frac x\e\right)$. Воспользуемся определениями свёртки и умножения обобщённой функции на бесконечно гладкую: $\forall\ \phi\in\D(\R)$
\[
  \big(f(x)g(y),\eta_k(x,y)\phi(x+y)\big) = \big(\tau_\e(x)f(x)\tau_\e(y)g(y),\eta_k(x,y)\phi(x+y)\big)=
  \big(f(x)g(y),\underbrace{\eta_k(x,y)\tau_\e(x)\tau_\e(y)\phi(x+y)}_{\in\D(\R^2)}\big)=\dots
\]
Почему просто понять, что у аргумента компактный носитель? Пусть $\supp\phi\subset[-A,A]$. Тогда $-A\le x+y\le A$. Теперь $\phi$ умножается на две функции, у которых носитель луч. А так как $\eta_k$ "--- компактное исчерпание единицы и существует предел это выражения (последовательность стабилизируется на компактном носителе произведения $\tau_\e(x)\tau_\e(y)\phi(x+y)$ с какого-то номера $k$.
\[
\dots = \big(f(x)g(y),\tau_\e(x)\tau_\e(y)\phi(x+y)\big)
\]
для достаточно больших $k$.

Таким образом, свёртка $f\star g$ существует, причём
\[
  \forall\ \phi\in\D(\R),\ \forall\ \e>0\pau (f\star g,\phi) = \big(f(x)g(y),\tau_\e(x)\tau_\e(y)\phi(x+y)\big).
\]
Осталось показать, что $\supp f\star g\subset [0,\infty)$. Пусть $\phi\in\D\big((-\infty,0)\big)$. Покажем, что $(f\star g,\phi)=0$. Для этого надо формулой воспользоваться.
\[
  \supp\phi\subset(-\infty,0)\imp\exists\ \e>0\colon \tau_\e(x)\tau_\e(y)\phi(x+y)\equiv0.
\]
% картинку надо нарисовать

Что является единицей? $\delta$-функия. Все функции с ней сворачиваются и результат свёртки сама функция.
\end{Proof}
